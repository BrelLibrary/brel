\section{Concepts}

In section \ref{sec:facts}, we learned that a fact is the smallest unit of information in an XBRL report. 
One of the core aspects of a fact is its concept. 
The concept refers to what is being reported. 
So for example, if a fact is reporting information about a company's revenue, then the concept of the fact is "Revenue".
In this section, we will take a closer look at concepts and how they are defined in XBRL.

Concepts are the fundamental building blocks of XBRL and they are defined in the XBRL taxonomy. 

\subsection{Taxonomy}

TODO: the root taxonomy should be called 'txtension taxonomy'

The XBRL taxonomy is a collection of concepts. 
Each XBRL report defines its own taxonomy inside of a taxonomy schema file.
This schema file is called the \textbf{root schema} of the taxonomy.
The root schema is the entry point to the taxonomy and it contains references to all other schemas in the taxonomy.
By resolving these references, the entire taxonomy can be loaded into memory.

The schemas being referenced include the following:

\begin{itemize}
    \item \textbf{us-gaap} \footnote[0]{https://xbrl.us/us-gaap/} - Contains concepts for US Generally Accepted Accounting Principles (GAAP).
    \item \textbf{ifrs} \footnote[1]{https://www.ifrs.org/} - Contains concepts for International Financial Reporting Standards (IFRS).
    \item \textbf{dei} \footnote[2]{https://www.sec.gov/info/edgar/dei-2019xbrl-taxonomy} - Contains concepts for the SEC's Document and Entity Information (DEI) requirements.
    \item \textbf{country} \footnote[3]{https://xbrl.fasb.org/us-gaap/2021/elts/us-gaap-country-2021-01-31.xsd} - Contains concepts for country codes.