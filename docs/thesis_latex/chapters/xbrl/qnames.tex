\section{QNames}

Although the motivation behind this XBRL processor is to shield its user from the complexity of XML, 
we keep one key aspect of XML in our API: QNames.

QNames are a way to uniquely identify an XML element or attribute. 
They consist of a local name and a namespace, which in turn consists of a namespace prefix and a namespace URI. 
The namespace URI is a URI that uniquely identifies the namespace and namespace prefix acts as a shorthand for the namespace.

For example the QName \texttt{us-gaap:Assets} identifies the element \texttt{Assets} in the namespace \texttt{us-gaap}.

In this example, the namespace prefix \texttt{us-gaap} is a shorthand for the namespace URI \texttt{https://xbrl.fasb.org/us-gaap/2022/elts/us-gaap-2022.xsd}, 
and together they form the namespace \texttt{us-gaap}.

\begin{figure}
    \caption{The us-gaap:Assets QName}
    \label{fig:qname_us_gaap_assets}
    \begin{itemize}
        \item Namespace prefix: \texttt{us-gaap}
        \item Namespace URI: \texttt{https://xbrl.fasb.org/us-gaap/2022/elts/us-gaap-2022.xsd}
        \item Local name: \texttt{Assets}
    \end{itemize}
\end{figure}

QNames are used in the XBRL taxonomy to identify concepts, facts and other elements. Since they provide a robust and easy way to identify elements,
we decided to use them in our API as well. However, there is one important difference between our QNames and the QNames used in the XBRL taxonomy:
In the XBRL taxonomy, the mapping from namespace prefixes to namespace URIs depends on where the QName is used. 
In our API, there is a fixed, global mapping from namespace prefixes to namespace URIs.

% Example of us-gaap and how it points to 2021 at one point and to 2022 at another point

% \hrule

% \textbf{Example 1:} The following XML document contains two \texttt{us-gaap:Assets} elements, each with a different namespace URI.

% \begin{lstlisting}
%     <?xml version="1.0" encoding="UTF-8"?>
%     <root xmlns:us-gaap="https://xbrl.fasb.org/us-gaap/2022/elts/us-gaap-2022.xsd">
    
%       <us-gaap:Assets>
%         <!-- Content for the first us-gaap:Assets element with XSD 2022 -->
%         <us-gaap:AssetValue>1000000</us-gaap:AssetValue>
%       </us-gaap:Assets>
    
%       <!-- Overwrite the namespace URI for the following element -->
%       <us-gaap:Assets xmlns:us-gaap="https://xbrl.fasb.org/us-gaap/2022/elts/us-gaap-2021.xsd">
%         <!-- Content for the second us-gaap:Assets element with XSD 2021 -->
%         <us-gaap:AssetValue>750000</us-gaap:AssetValue>
%       </us-gaap:Assets>
    
%     </root>
% \end{lstlisting}

% In this example, the first \texttt{us-gaap:Assets} element maps to the 2022 version of the US-GAAP taxonomy, while the second \texttt{us-gaap:Assets} element maps to the 2021 version of the US-GAAP taxonomy.
% This is possible because the namespace prefix \texttt{us-gaap} is redefined in the second \texttt{us-gaap:Assets} element.

% \hrule

% \textbf{Example 2:} Example that defines the same concept twice, once with 2021 and once with 2022.
