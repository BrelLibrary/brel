\chapter{Introduction}
\label{sec:introduction}
%\label{sec:einleitung}

\section{Introduction}

% Write an introduction to your thesis here.

In the era of data-driven decision-making, the ability to efficiently interpret and analyze business reports is becoming increasingly important.
% Until around 20 years ago, most business reports were published in paper form.
% This made it difficult for computers to automatically process the information contained in the reports.
% However, this changed with the introduction of the eXtensible Business Reporting Language (XBRL) in 2000\cite{aicpa_xbrl_story}\footnote{Marks the release of XBRL 1.0}.
% XBRL is a standardized format for business reports that is both machine-readable and can produce human-readable reports.
% Originally designed for financial reports, XBRL is now used in a wide variety of business reports.\cite{xbrl_about}
% Initially, XBRL was based on XML, but it has since been extended to support other formats such as JSON and CSV.
% In the contemporary landscape of data-driven decision making,
% the proficiency in interpreting and analyzing business reports has become increasingly pivotal.
Approximately 20 years ago,
the predominant medium for publishing business reports was paper.
% This format posed challenges for the automated processing of the information encapsulated in the reports by computers.
This format posed challenges for automated processing of reports by computers. 
Nevertheless,
the advent of the eXtensible Business Reporting Language (XBRL) in the year 2000 marked a significant shift in this domain\cite{aicpa_xbrl_story}.
% XBRL offers a standardized schema for business reports,
% facilitating both machine-readability and the generation of reports comprehensible by humans.
XBRL is a standardized format for business reports that is both machine-readable and can produce human-readable reports.
Initially conceptualized for financial reporting,
% the application of XBRL has expanded to encompass a broad spectrum of business reporting contexts.\cite{xbrl_about}
XBRL is now used in a wide variety of business reports\cite{xbrl_about}.
% While the foundation of XBRL was on XML,
While early iterations of XBRL were exclusively based on XML, 
subsequent developments have enabled its compatibility with additional formats such as JSON and CSV.

% Back in the day, XBRL was a niche technology that was only used by a handful of companies.
% Now, XBRL is gaining traction in both the public and private sectors.
% Reporting authorities such as the US Securities and Exchange Commission (SEC) and the European Banking Authority (EBA)\cite{eba_reporting_frameworks} are increasingly requiring companies to submit their reports in XBRL format.\cite{sec_ixbrl}
% In the private sector, XBRL is used by companies such as JP Morgan Chase, Microsoft, and Hitachi to automate their financial reporting processes.\cite{pwc2002thejournal}
Previously, XBRL was a specialized technology utilized by a select group of companies.
Presently, XBRL is witnessing increased adoption across both public and private sectors.
% Regulatory bodies such as the US Securities and Exchange Commission (SEC)\cite{sec_ixbrl} and the European Banking Authority (EBA)\cite{eba_reporting_frameworks} are progressively mandating the submission of reports in XBRL format.
Both the US Securities and Exchange Commission (SEC)\cite{sec_ixbrl} and the European Banking Authority (EBA)\cite{eba_reporting_frameworks} are increasingly requiring companies to submit their reports in XBRL format.
In the corporate domain, entities like JP Morgan Chase, Microsoft, and Hitachi are leveraging XBRL to streamline their financial reporting mechanisms.\cite{pwc2002thejournal}

% However, XBRL is not without its problems.
% As a standard, XBRL is very complex.
% Many of its intricacies are a result of legacy design decisions and make working with XBRL unnecessarily difficult.
% % On top of that XBRLs design is closely tied to the XML format, even though standard is not strictly tied to XML anymore.
% In addition, XBRL is closely tied to the XML format, even though the standard is not strictly tied to XML anymore.
% Nonetheless, XBRL encompasses certain complexities.
% As a framework, XBRL embodies considerable intricacy,
% with many complexities stemming from historical design choices that contribute to its challenging usability.
Nonetheless, XBRL encompasses certain complexities,
many of which are stemming from legacy design decisions.
% Moreover, despite XBRL's evolution beyond its XML roots,
% the association with XML format persists.
Much of XBRL's design is closely tied to the XML format,
despite the standard's departure from XML exclusivity.

Given XBRL's primary audience of non-technical users, its accessibility is crucial.
% Yet, the usability of XBRL in its current form is not optimal.
% The landscape of XBRL tools is diverse, with a prevalence of proprietary and closed-source solutions.
% While open-source alternatives are available, they are limited in number.
% With the exception of the XBRL platform Arelle\cite{arelle}, these tools are often either underdeveloped or inadequately documented.
% Given that XBRL is primarily tailored for users without a technical background,
% its usability is of utmost importance.
% Nevertheless,
% the present status of XBRL does not fully align with this requirement.
% This gave rise to a diverse landscape of XBRL tools,
% % The XBRL ecosystem encompasses a diverse array of tools,
% % The XBRL landscape is characterized by a diverse array of tools,
% with a majority being proprietary and not openly accessible.
% While there are open-source alternatives available,
% they are limited in number.
% Apart from Arelle, a notable platform in the XBRL domain\cite{arelle},
% other open-source are often limited in scope.
% Other notable open-source options are \texttt{python-xbrl}\cite{git_python_xbrl}, \texttt{pysec}\cite{git_pysec}, and \texttt{py-xbrl}\cite{git_py_xbrl}.
% None of these libraries support the OIM.
However, the current state of XBRL does not entirely meet this requirement.
This situation has led to the emergence of a varied range of XBRL tools,
most of which are proprietary and not freely available.
Although there are some open-source alternatives,
% they are relatively few in number and often limited in scope.
they are often limited in scope.
One significant exception is Arelle\cite{arelle}, a well-known tool in the XBRL field.

% Other open-source python libraries, like \texttt{python-xbrl}\cite{git_python_xbrl}, \texttt{pysec}\cite{git_pysec}, and \texttt{py-xbrl}\cite{git_py_xbrl}, 
% often have a limited range of functions and do not support the latest XBRL features.
% are often limited in scope and do not support XBRL's latest key feature.
% Notably, none of these libraries apart from Arelle support the latest XBRL features.

% In 2021, XBRL International published a partial rewrite of the XBRL standard called Open Information Model (OIM)\cite{oim}.
% The goal of OIM is to learn from the mistakes of the past and to make XBRL more accessible to both developers and non-developers.
% In fact, the OIM is not a specification at all.
% It is merely a syntax independent conceptual model for XBRL reports.
% Therefore, compared to the original XBRL standard, the OIM is not strictly tied to the XML format.
% The OIM is a major step in the right direction, but it is still in its infancy and only reworks some of the core XBRL concepts.
% Also, at the time of writing, the design of the OIM is not yet finalized.

In 2021, XBRL International published a new specification termed Open Information Model (OIM)\cite{oim}.
The OIM is a logical data model for XBRL reports that is independent of the XBRL syntax.
% The OIM iterates on the XBRL standard and aims to make XBRL more accessible to both developers and non-developers.
One objective of the OIM is to make XBRL more approachable for both developers and non-technical users by iterating on XBRL's design.
% The OIM is a step in the direction of syntactic independence, but at the time of writing, 
The OIM is not yet finalized and does not cover all aspects of XBRL. 

% In 2021, XBRL International introduced a new specification termed the Open Information Model (OIM)\cite{oim}.
% The OIM acts as a logical data model for XBRL reports 
% and is characterized by its independence from the conventional XBRL syntax.
% The primary objective of the OIM is to enhance the usability of XBRL, making it more approachable for both developers and non-technical users.
% The OIM represents a stride towards achieving syntactic versatility within the XBRL framework.
% However, it is important to note that, at the current stage, the structure of the OIM is not entirely comprehensive,
% and it does not encapsulate all facets of XBRL.
% The design of the OIM is not yet finalized and does not cover all aspects of XBRL. 

\section{Goals of this thesis}
\label{sec:goals}

% The goal of this thesis is to create a new open source XBRL library that is based on the XBRL standard, notably the OIM.
This goal of this thesis is to develop an open source XBRL library based on the XBRL standard, particularly the OIM.
% The name of the library is \texttt{Brel} (short for Business Reporting Extensible Library) and it is written in the Python programming language.
% The library should be easy to use and well documented.
% The library is called \texttt{Brel} (short for Business Reporting Extensible Library) and is written in the Python programming language and should be easy to use.
The library, named \texttt{Brel} (short for Business Reporting Extensible Library), is written in the Python programming language.
% It is designed to be easy to use and well documented. should be easy to use and well documented.
% It should provide a simple python API that allows developers to easily read XBRL reports and extract information from them.
Brel should provide a simple python API that allows developers to easily read XBRL reports and extract information from them.
Fundamentally, the library should act as a pythonic wrapper around all elements of an XBRL report.
% The library should also provide basic functionality for validating XBRL reports.
% Finally, the library should support XBRL reports in XML, but it should be designed in a way that allows it to support other formats in the future.
Lastly, the library should support XBRL reports in XML, but its design should be extensible to support other formats in the future.

% Brel is distinct from Arelle.  
Even though Brel and Arelle both target the XBRL domain, they are distinct from each other.
While Arelle is a full platform for XBRL, similar to Excel for spreadsheets,  
Brel is a Python library, integrating smoothly with Python's ecosystem.

The research questions that this thesis aims to answer are:


\begin{itemize}
    \item \textbf{\customlabel{RQ1}{RQ1}} How can the OIM be translated into a Python API?
    \item \textbf{\customlabel{RQ2}{RQ2}} How can the non-OIM sections of XBRL be converted into a Python API that is consistent with the OIM?
    \item \textbf{\customlabel{RQ3}{RQ3}} How can the library be designed to support multiple formats in the future?
\end{itemize}

\section{Limitations of this thesis}
\label{sec:limitations}

% The XBRL standard is very complex, too complex to be covered in its entirety in a single thesis.
% Since its inception in 2000, the standard did not only grow in size, but also in complexity.
The XBRL standard has grown in size and complexity since its inception in 2000.
% In its current form, implementing a complete XBRL library is not feasible within the scope of a single thesis.
In its current form, implementing a complete XBRL library is not feasible within the scope of a single thesis.
% It has grown to a point where it is no longer feasible to implement a complete XBRL library within the scope of a single thesis.
% Therefore, the implementation of Brel will have to make some compromises.
Therefore, Brel will have to make some compromises.

% The first two limitations of this thesis are therefore, the library will only support XBRL reports in XML format.
The first limitation of this thesis is that the library will only support XBRL reports in XML format.
Secondly, the library will only support reading XBRL reports, not creating or modifying them.
% The library will also only support reading XBRL reports, not creating or modifying them.
% The third limitation of Brel is a matter of report semantics.
Third, Brel will not semantically validate XBRL reports.

% Whereas the first two limitations are self-explanatory, the third limitation requires some explanation.
While the initial two limitations are straightforward, the third limitation necessitates further clarification.
% Similar to how the source code of a program can be interpreted in terms of syntax and semantics, XBRL reports can be viewed through the same lens.
XBRL reports can be interpreted in terms of syntax and semantics, similar to the source code of a program.
% A syntactically correct XBRL report conforms to the XBRL specification, but it does not necessarily make sense
An XBRL report that is syntactically correct adheres to the XBRL specifications, yet it may not necessarily be logically coherent\footnote{The XBRL specification does sometimes branch out into the realm of semantics. Brel ignores these parts of the specification.}.
A semantically correct report is both syntactically correct and logically coherent.
% Semantical correctness started bleeding into the XBRL specification through supporting documents.
% This thesis will not cover the semantics of XBRL reports.
% This thesis will not cover the semantics of XBRL reports, which are not strictly defined by the XBRL specification, but rather through supporting documents.
This thesis will not address the semantics of XBRL reports, 
as they are rarely detailed in the XBRL specification,
%  but are outlined in supplementary documents.
but are instead outlined in supporting documents.

% Through the Brel API, it is possible to semantically validate XBRL reports to be in line with the XBRL supporting documents.
% By using the Brel API, it is possible to semantically validate XBRL reports.
% However, by using the Brel API, one can semantically validate XBRL reports.
Nonetheless, semantic validation of XBRL reports can be achieved through the use of the Brel API.
% The functionality for semantic validation is not part of the Brel library, but can be implemented on top of the Brel API.
% The reader can think of Brel as a syntactic wrapper around XBRL reports.
% However, this thesis will not cover the semantics of XBRL reports in order to keep the scope of the thesis manageable.
% At this point, think of Brel as a syntactic wrapper around XBRL reports.
% The only exception to this rule is in calculation networks, which will be explained in chapter \ref{chapter:xbrl}.
% Think of them as a proof of concept for how Brel could be extended to support semantic validation in the future.
% Some of the limitations discussed in this section will be addressed in chapter \ref{chapter:conclusion}.

\section{Structure of this thesis}

% Since this thesis is about XBRL, it is important to first understand what XBRL is and how it works.
% To properly understand how Brel implements a pythonic API for XBRL reports, it is important to first understand the underlying XBRL standard.
Grasping the underlying XBRL standard is essential to understand how Brel provides a pythonic API for XBRL reports.
% Therefore, chapter \ref{chapter:xbrl} will give a brief introduction to XBRL. 
Hence, Chapter \ref{chapter:xbrl} will offer a concise introduction to XBRL.
% It will introduce the core concepts of XBRL in the OIM and move on to the non-OIM sections of XBRL.
% It will introduce the core concepts of XBRL in the OIM, followed by the non-OIM sections of XBRL.
The chapter will present the fundamental concepts of XBRL within the framework of the OIM, followed by an exploration of the non-OIM aspects of XBRL.
% It will not dive deep into the technical details of the XBRL standard, but instead focus on the concepts that are relevant for this thesis.
% The chapter will not delve into the technical details of the XBRL standard, but rather focus on the concepts relevant for this thesis.
This chapter will focus on the concepts relevant to this thesis, rather than delving deep into the technical specifics of the XBRL standard.

Subsequently, Chapter \ref{chapter:api} will introduce the API of Brel.
% Next, chapter \ref{chapter:api} will introduce the API of Brel.
% This chapter is already part of the result of this thesis, but it makes more sense to introduce the API before its implementation.
% This chapter, while constituting a portion of the thesis outcomes, is positioned prior to the implementation of the API.
This chapter, which forms part of the thesis results, is positioned prior to the implementation of the API.
The reason for this deviation from the conventional structure is that it is more logical to introduce the API before its implementation.
The API chapter will answer research questions \ref{RQ1} and \ref{RQ2}.

% Chapter \ref{sec:implementation} details the implementation of the Brel library, 
% linking the API described in chapter \ref{chapter:api} with the XBRL standard outlined in chapter \ref{chapter:xbrl}. 
% % Due to Brel's extensive and complex structure, this chapter will only focus on key aspects of its implementation. 
% % While most XBRL to Python translations are straightforward, certain cases are more complex. 
% % The chapter will focus primarily on the latter, and explain how Brel handles them. 
% Even though this chapter aims to provide a comprehensive overview of Brel's implementation,
% it will primarily focus on the aspects of the implementation which are more involved.
% The chapter will also discuss the library's design, specifically its capability to support various formats in future versions, 
% addressing Research Question \ref{RQ3}.
% Chapter \ref{sec:implementation} outlines the Brel API's implementation,
% connecting the API from Chapter \ref{chapter:api} with the XBRL standards in Chapter \ref{chapter:xbrl}.
Chapter \ref{sec:implementation} details the implementation of the Brel API, 
linking the API discussed in Chapter \ref{chapter:api} with the XBRL standards explored in Chapter \ref{chapter:xbrl}.
% While aiming for a comprehensive overview,
% the chapter will particularly emphasize the more involved aspects of the implementation.
While aiming for a comprehensive overview,
the chapter will particularly emphasize the more involved aspects of the implementation.
% It will also discuss the library's design and its potential to support various formats in future versions,
% addressing Research Question \ref{RQ3}.
Additionally, the chapter will explore the design of the library and its capability to accommodate different formats in future iterations, 
thereby addressing Research Question \ref{RQ3}.

% The Results chapter \ref{chapter:results} will evaluate the library based on the goals set out in the introduction.
% This chapter will also give a comprehensive overview of how Brel performs in terms of various XBRL conformance suites.
% Additionally, it will give examples of how the library can be used to read and validate XBRL reports.
% % In the se results chapter \ref{chapter:results}, the library will be evaluated based on the goals set out in the introduction.
% % This chapter will also give a comprehensive overview of the library's features and limitations.
% % Additionally, it will give examples of how the library can be used to read and validate XBRL reports.

% Lastly, chapter \ref{chapter:conclusion} will summarize the results of this thesis and give an outlook on future work.
% Chapter \ref{chapter:results}, the Results chapter, assesses the library's performance against the objectives established in the introduction.
% This chapter provides a detailed analysis of Brel's compatibility with various XBRL conformance suites.
% It also illustrates practical applications of the library, demonstrating how Brel can be utilized for reading and validating XBRL reports.
Chapter \ref{chapter:results}, known as the Results chapter, 
evaluates the library's effectiveness in meeting the goals set out in the introduction. 
% This chapter offers an in-depth examination of Brel's alignment with different XBRL conformance suites. 
This chapter examines Brel's alignment with an XBRL conformance suite.
It will evaluate the library based on its performance compared to Arelle, 
as well as its robustness in handling various XBRL reports.
% Brel differs from Arelle, as Arelle is a platform for XBRL reports akin to Excel for spreadsheets.
% Brel, on the other hand, is a python library that seamlessly integrates into the python ecosystem.
Additionally, it showcases real-world uses of the library, 
highlighting how Brel can be employed for reading and verifying XBRL reports.

Chapter \ref{chapter:conclusion}, the concluding chapter,
will reiterate the main discoveries of this thesis.
It will also offer insights into possible avenues for further research and development related to Brel.