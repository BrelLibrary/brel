\section{Answering research question 3}
\label{sec:answer_research_question_3}

With the implementation of Brel complete for the XBRL XML syntax, we can now answer research question \ref{itm:research_question_3}:

\begin{displayquote}
    \textbf{RQ3:} How can the library be designed to support multiple formats in the future?
\end{displayquote}

The answer to this question is that the Brel API is designed to be almost format agnostic.
The first half of the Brel API is based on the OIM, which in itself is a logical data model.
This makes it inherently format agnostic.

The second half of the Brel API is based on the XBRL XML syntax.
However, the only methods that still expose the XBRL XML syntax are the methods
\texttt{get\_link\_role} and \texttt{get\_link\_name} in the \texttt{INetwork} interface 
as well as the \texttt{get\_arc\_role} and \texttt{get\_arc\_name} methods in the \texttt{INetworkNode} interface.
These methods return the XML attributes of the same name.
However, the main purpose of these methods is debugging.

The only real remnant of the XBRL XML syntax in the Brel API is the \texttt{QName} class.
However, Brel merely chooses to use the same data structure as QNames in XML, 
which is a combination of a prefix, a namespace URI and a local name.
The Brel API does not require QNames to be in the XML format.
In fact, both the JSON\cite{xbrl_json} and the CSV\cite{xbrl_csv} specifications of XBRL use QNames in the same format as the XML specification.
Therefore, the \texttt{QName} class is also format agnostic.

Since the Brel API is format agnostic, the only section of Brel that relies on the XBRL XML syntax is the parser.
Brel implements the parser as a separate module, which is called \texttt{brel.parser.XML}.
The parser module is the only module that needs to be changed in order to support other formats.
The rest of Brel can remain unchanged.
