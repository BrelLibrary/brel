% \section{Answering research question 3}
% \label{sec:answer_research_question_3}

% With the implementation of Brel complete for the XBRL XML syntax, we can now answer research question \ref{RQ3}:

% \begin{displayquote}
%     \textbf{RQ3:} How can the library be designed to support multiple formats in the future?
% \end{displayquote}

% The answer to this question is that the Brel API is designed to be almost format agnostic.
% The first half of the Brel API is based on the OIM, which in itself is a logical data model.
% This makes it inherently format agnostic.

% The second half of the Brel API is based on the XBRL XML syntax.
% However, the only methods that still expose the XBRL XML syntax are the methods
% \texttt{get\_link\_role} and \texttt{get\_link\_name} in the \texttt{INetwork} interface 
% as well as the \texttt{get\_arc\_role} and \texttt{get\_arc\_name} methods in the \texttt{INetworkNode} interface.
% These methods return the XML attributes of the same name.
% However, the main purpose of these methods is debugging.

% The only real remnant of the XBRL XML syntax in the Brel API is the \texttt{QName} class.
% However, Brel merely chooses to use the same data structure as QNames in XML, 
% which is a combination of a prefix, a namespace URI and a local name.
% The Brel API does not require QNames to be in the XML format.
% In fact, both the JSON\cite{xbrl_json} and the CSV\cite{xbrl_csv} specifications of XBRL use QNames in the same format as the XML specification.
% Therefore, the \texttt{QName} class is also format agnostic.

% Since the Brel API is format agnostic, the only section of Brel that relies on the XBRL XML syntax is the parser.
% Brel implements the parser as a separate module, which is called \texttt{brel.parser.XML}.
% The parser module is the only module that needs to be changed in order to support other formats.
% The rest of Brel can remain unchanged.

\section{Addressing Research Question 3}
\label{sec:answer_research_question_3}

Now that Brel's implementation for the XBRL XML syntax is complete, we can address research question \ref{RQ3}:

\begin{displayquote}
    \textbf{RQ3:} How can the library be designed to accommodate multiple formats in the future?
\end{displayquote}

To make Brel compatible with multiple formats, the design of the Brel API need to be format-agnostic. 
The design of the Brel API is largely format-independent. 
% Its first segment is grounded in the OIM, which is inherently a logical data model, contributing to its format-agnostic nature.
Its first segment is grounded in the OIM, which is a logical data model and thus inherently format-agnostic.
The latter half of the Brel API, while based on the XBRL XML syntax, largely abstracts away the specifics of this format. 

% The only exceptions are the \texttt{get\_link\_role} and \texttt{get\_link\_name} methods in the \texttt{INetwork} interface, 
% as well as the \texttt{get\_arc\_role} and \texttt{get\_arc\_name} methods in the \texttt{INetworkNode} interface. 
The only exceptions are the \texttt{get\_link\_role}, \texttt{get\_link\_name}, \texttt{get\_arc\_role}, and \texttt{get\_arc\_name} methods.
These methods return attributes bearing the same names in XML, primarily serving debugging purposes.
They are not essential to the API's functionality.
% Therefore, the se Brel API is almost entirely format-agnostic, with the exception 
Therefore, the second half of the Brel API is almost entirely format-agnostic, with the exception of debugging methods.

% The main aspect of the XBRL XML syntax within the Brel API is the \texttt{QName} class. 
The primary aspect where Brel relies on the XBRL XML syntax is the \texttt{QName} class.
This class mirrors the QName structure in XML, comprising a prefix, a namespace URI, and a local name. 
% However, the API's use of QNames is not restricted to XML formatting. 
However, even though QNames originate from XML, they have been adopted in other XBRL specifications.
Notably, both the JSON\cite{xbrl_json} and CSV\cite{xbrl_csv} specifications of XBRL adopt QNames in a similar structure to the XML specification, 
rendering the \texttt{QName} class format-neutral.

Given the API's format-agnostic design, the aspect of Brel that relies on the XBRL XML syntax is exclusively the parser. 
% Brel's parser is encapsulated in a distinct module, named \texttt{brel.parser.XML}. 
The parser outlined in this chapter is encapsulated in a distinct module, named \texttt{brel.parser.XML}
To support different formats, only this parser module needs modification, allowing the rest of Brel to remain as is.
Thus, the Brel API is designed to be adaptable to future XBRL formats.
