\section{Implementation of Networks}

\subsection{Overview}

In chapter \ref{chapter:xbrl}, we introduced the concept of networks.
We also looked at the different types of networks that are defined in the XBRL 2.1 specification.
In this chapter, we will look at how networks are represented in XBRL and how Brel parses them.

As previously mentioned, one of the main goals of Brel is to shield the user from the complexity of the XBRL specification.
Brel achieves this on two different levels.

First, Brel implements a wrapper around each type of network.
This wrapper provides a clean interface for some of the most common operations on networks.
The functionality of the wrapper depends on the type of network.

For example, the wrapper for calculation networks provides functions to validate the calculation network and to calculate the value of a concept.

Second, Brel provides a simple API that allows users to access and traverse networks independent of the type of network.
This API provides methods for traversing any directed acyclic graph.
It is intended for advanced users and is useful for debugging networks in a report.

For example, for each network, Brel provides a function that returns all direct children of a node in the network.


