\section{General Implementation}
\label{sec:implementation_general}

% Brel parses XBRL reports using an eager bottom-up approach
% It starts with the smallest building blocks of XBRL reports - report elements.
% After all report elements have been parsed, Brel moves on to parsing facts and their associated characteristics.
% Next, Brel parses all networks and their associated resources.
% Finally, Brel parses the components of the report.

% Brel chooses this bottom-up approach because both networks and facts depend on report elements.
% Networks depend on report elements since their nodes can point to report elements.
% Facts depend on report elements since their characteristics can refer to concepts, dimensions and members.
% Networks and facts often refer to the same report elements.
% Therefore, their python classes should share the same report element instances.
% The bottom-up approach ensures that all report elements are parsed before they are used by networks and facts.
% In the next four sections, this chapter will briefly cover the four stages of Brel's bottom-up approach.
Brel systematically processes XBRL reports using an eager bottom-up strategy.
The process begins with the fundamental units of XBRL reports - the report elements.
Once every report element is parsed, Brel progresses to interpreting facts and their related characteristics.
Subsequently, Brel examines all networks along with their connected resources.
Finally, Brel analyzes the components of the report.

The rationale for this bottom-up method is the reliance of both networks and facts on report elements.
Networks require report elements as their nodes may link to these elements.
Similarly, facts need report elements because their attributes can be associated with concepts, dimensions, and members.
It is common for both networks and facts to reference identical report elements.
Hence, their corresponding Python classes should utilize the same instances of report elements.
Adopting a bottom-up approach guarantees that all report elements are fully parsed prior to their utilization in networks and facts.
The subsequent four sections of this chapter will briefly discuss each phase of Brel's bottom-up parsing approach.

\subsection{Parsing Report Elements}
\label{sec:implementation_report_elements}

% Report elements are the smallest building blocks of XBRL reports.
% Therefore, they do not rely on any other XBRL elements and can be parsed first.
% Report elements are defined in the taxonomy set of the XBRL report, which is a collection of \texttt{.xsd} files in the XML format.
% These files are all stored locally on the user's computer.
% Even though XBRL does not require the taxonomy set to be stored locally, Brel does.
% However, Brel automatically downloads the taxonomy set from the internet if it is not already stored locally.
% The mechanism for downloading the taxonomy set is discussed in section \ref{sec:implementation_dts_caching}.

% Taxonomies contain three types of elements - linkbases, roles and report elements.
% Linkbases are discussed in section \ref{sec:implementation_networks}.
% Roles are discussed in section \ref{sec:implementation_components}.
% Report elements are discussed in this section.

% In XBRL, a taxonomy can refer to other taxonomies and associate them with a namespace prefix.
% % In most instances, different taxonomies agree on which namespace prefix and URI a given taxonomy should use.
% For now, the reader can assume that different taxonomies agree on which namespace prefix and URI a given taxonomy should use.
% Brel ensures that this assumption holds true in a process called namespace normalization, which is discussed in section \ref{sec:implementation_namespace_normalization}.
% If all taxonomies agree on a prefix and URI for a given taxonomy, 
% then all report elements defined in that taxonomy inherit the same prefix and URI as part of their QName.

% Report elements within a taxonomy are organized in a flat list of XML elements
% Each XML element has a unique name attribute, which represents the local name of the report element's QName.
% Since the Brel API lists six different types of report elements, Brel needs to decide which type of report element each XML element represents.
% There is no single attribute in the XML element that indicates the type of report element.
% Instead, Brel uses a combination of different attributes to determine the type of report element.
% The process of determining the type of report element is outlined in the following table:

Report elements represent the most fundamental components of XBRL reports.
% As such, they are independent of other XBRL elements and are parsed initially.
As such, they do not rely on other XBRL elements and are parsed first.
These elements are specified in the XBRL report's taxonomy set, which consists of a series of \texttt{.xsd} files in XML format.
% All these files reside locally on the user's computer.
For now the reader can assume that all files are stored locally on the user's computer.
While XBRL does not mandate local storage of the taxonomy set, Brel requires it.
However, Brel is designed to automatically download the taxonomy set from the internet if it is not already available locally.
Details about this downloading process are in section \ref{sec:implementation_dts_cache}.

Taxonomies comprise three element types: linkbases, roles, and report elements.
Linkbases are covered in section \ref{sec:implementation_networks}.
Roles are addressed in section \ref{sec:implementation_components}.
This section focuses on report elements.

In XBRL, a taxonomy can reference other taxonomies and assign them a namespace prefix.
For the moment, it is assumed that different taxonomies concur on the namespace prefix and URI for a given taxonomy.
Brel validates this assumption through a procedure known as namespace normalization, discussed in section \ref{sec:qnames_implementation}.
When there is consensus on a prefix and URI for a taxonomy, all report elements defined within inherit the same prefix and URI as part of their QName.

Within a taxonomy, report elements are arranged as a flat list of XML elements.
Each XML element is uniquely identified by a name attribute, which denotes the local name of the report element's QName.
Given that the Brel API identifies six distinct types of report elements, Brel must determine the specific type for each XML element.
This decision is not based on a single attribute in the XML element but rather on a combination of various attributes.
The methodology used to ascertain the type of each report element is detailed in the following table:

\begin{table}[H]
    \centering
    \scalebox{0.85}{
    \begin{tabular}{|l|l|l|l|l|}
        \hline
        \makecell[l]{\textbf{Report element} \\ \textbf{type}}  & 
        % \textbf{priority} & 
        % \textbf{XML abstract attribute} &
        \makecell[l]{\textbf{XML abstract} \\ \textbf{attribute}} &
        \makecell[l]{\textbf{XML substitutionGroup} \\ \textbf{attribute}} & 
        % \textbf{XML type attribute} \\ \hline
        \makecell[l]{\textbf{XML type} \\ \textbf{attribute}} \\ \hline
        Member      &           & "xbrli:item"                          & "domainItemType" \\ \hline
        Concept     & "false"   &                                       &                            \\ \hline
        Hypercube   & "true"    & "xbrldt:hypercubeItem"                &                            \\ \hline
        Dimension   & "true"    & "xbrldt:dimensionItem"                &                            \\ \hline
        Abstract    & "true"    & "xbrli:item"                          &                            \\ \hline
    \end{tabular}
    }
    \caption{Determining the type of report element}
    \label{tab:determining_report_element_type}
\end{table}

% Brel uses the table \ref{tab:determining_report_element_type} above to determine the type of report element.
% It traverses the table from top to bottom and selects the first row where all conditions are met.
% If a cell in the table is empty, then Brel ignores that condition. 

% The table above does not contain a row for the type "LineItems".
% The reason for this is that line items and abstracts can not be distinguished by their XML attributes.
% They can only be distinguished by their position within a definition network.
% Therefore, Brel parses line items and abstracts as abstracts.
% Brel later uses the definition network to determine which abstracts are line items, and fixes the type of these abstracts accordingly.
% This process is discussed in section \ref{sec:implementation_networks}.

% Once all report elements have been parsed, Brel creates a lookup table for report elements.
% Given a QName, the lookup table returns the corresponding report element instance.
% This lookup table is used throughout the rest of the parsing process.

% Brel employs the table \ref{tab:determining_report_element_type} to identify the type of each report element.
Brel implements the procedure outlined in table \ref{tab:determining_report_element_type} to identify the type of each report element.
It examines the table from the top to the bottom, choosing the first row that fulfills all the specified conditions.
If a cell in the table is blank, Brel disregards that particular condition.

The table does not include a row for the "LineItems" type.
This is because line items and abstracts are indistinguishable based on their XML attributes alone.
They can be differentiated only through their placement within a definition network.
As a result, Brel initially categorizes both line items and abstracts as abstracts.
Later, within the context of the definition network, Brel determines which abstracts are actually line items and adjusts their types accordingly.
This procedure is further elaborated in section \ref{sec:implementation_networks}.

After parsing all report elements, Brel establishes a lookup table for these elements.
This table, when provided with a QName, returns the corresponding instance of the report element.
Brel utilizes this lookup table extensively in the subsequent stages of the parsing process.

% As mentioned in section \ref{sec:hypercubes}, the current implementation of Brel does not support open hypercubes.
% This is a feature that is planned for the future.
% The reason for this is that open hypercubes can reference report elements that are not defined in the taxonomy.
% Since Brel's parsing process is bottom-up, it requires all report elements to be defined in the taxonomy before it can parse networks and facts.
% This is in direct conflict with the concept of open hypercubes.
% In future versions of Brel, the parsing process will be restructured to support open hypercubes.

\subsection{Parsing Facts}

% Brel parses facts directly after parsing report elements.
% Facts are parsed before networks because footnote networks can point to facts.

% Facts are exclusively defined in the instance document of the XBRL report.
% The instance document is an XML which contains a flat list of facts, syntactic contexts and units, which are represented by XML elements.
% It may also contain a list of footnotes, which are discussed in section \ref{sec:implementation_networks}.

% \textbf{Fact} XML elements contain the value of the fact as well as references to the syntactic context and unit.
% The tag of the XML element is the QName of the concept of the fact. 

% \textbf{Syntactic context} XML elements describe a subset of the characteristics of a fact.
% They are different from \texttt{Context}s as defined by the Brel API.
% A \texttt{Context} in Brel contains all the characteristics of a fact, whereas a syntactic context only contains the period, entity and dimensions of a fact.
% During parsing, Brel uses syntactic contexts as a starting point for creating \texttt{Context} instances.
% It then adds the remaining characteristics - the concept and the unit - to the \texttt{Context} instance.

% \textbf{Unit} XML elements define, as the name suggests, the unit of a fact.

% The reason why XBRL separates facts, syntactic contexts and units into three different XML elements is to reduce redundancy.
% Facts can share the same syntactic context and unit.

% Brel parses all facts by finding all fact XML elements and resolving their references to syntactic contexts and units.
% It re-uses units, entities and dimensions across different facts. 

Brel processes facts immediately following the parsing of report elements.
Facts are analyzed prior to networks because footnote networks may reference facts.

Facts are solely defined in the instance document of the XBRL report.
This document is an XML file containing a straightforward list of facts, syntactic contexts, and units, represented as XML elements.
It might also include a list of footnotes, which are detailed in section \ref{sec:implementation_networks}.

\textbf{Fact} XML elements hold the fact's value and references to both the syntactic context and unit.
The XML element's tag represents the QName of the concept associated with the fact.

\textbf{Syntactic context} XML elements outline a part of a fact's characteristics.
They differ from \texttt{Contexts} as defined in the Brel API.
A \texttt{Context} in Brel encompasses all characteristics of a fact, while a syntactic context includes only the period, entity, and dimensions of a fact.
During parsing, Brel initially uses syntactic contexts to create \texttt{Context} instances.
Subsequently, it supplements the \texttt{Context} with the remaining characteristics - the concept and the unit.

\textbf{Unit} XML elements, as their name implies, define the unit of a fact.

The rationale for XBRL segregating facts, syntactic contexts, and units into distinct XML elements is to minimize redundancy.
Multiple facts can share the same syntactic context and unit.

Brel parses all facts by identifying all fact XML elements and resolving their links to syntactic contexts and units.
It reutilizes units, entities, and dimensions across various facts.

\subsection{Parsing Components}
\label{sec:implementation_components}

Components represent the final aspect of the XBRL report that Brel parses.
By this stage, Brel has already processed all report elements, facts, and networks.
% This chapter has not yet delved into networks due to their complexity, which is addressed in a separate section, section \ref{sec:implementation_networks}.
This chapter has not yet discussed networks due to their complexity, which are addressed in section \ref{sec:implementation_networks}.
% For the moment, it is assumed that Brel has successfully parsed all networks and a network lookup table is in place.
For the moment, the reader can assume that Brel has successfully parsed all networks and a network lookup table is in place.

Components, akin to report elements, are specified in the XBRL report's taxonomy set.
In XBRL terminology, these are referred to as "roleTypes" rather than "components".
To parse all components, Brel examines every taxonomy file for roleType XML elements.
These roleType XML elements encompass three elements: a role URI, an optional description, and a list of used-on elements.
\texttt{Component}s in Brel directly extract both the role URI and the description from the roleType XML element.
To identify the networks associated with a component, Brel searches the network lookup table using the role URI.

The used-on elements denote a list of network types authorized to utilize the component.
For instance, if the network lookup yields a \texttt{PresentationNetwork} instance, the roleType XML element must include "presentationLink" in its used-on elements list.

This segment concludes the discussion on Brel's general implementation.
Excluding networks, this section has encompassed every aspect of XBRL and Brel's method of parsing it.
The ensuing section will delve into the intricacies of network parsing.

% Components are the last section of the XBRL report that Brel parses.
% Before parsing components, Brel has already parsed all report elements, facts and networks.
% So far, this chapter has not discussed networks.
% Since parsing networks is complex, it is discussed in its own section - section \ref{sec:implementation_networks}.
% For now, the reader can assume that Brel has already parsed all networks, and that there is a lookup table for networks.

% Components, like report elements, are defined in the taxonomy set of the XBRL report.
% XBRL chooses to call them "roleTypes" instead of "components".
% To parse all components, Brel scans all taxonomy files for roleType XML elements.
% The roleType XML elements contain a three elements - A role URI, an optional description and a list of used-on elements.
% \texttt{Component}s in Brel directly read both the role URI and the description from the roleType XML element.
% To get the networks that are associated with a component, Brel indexes the lookup table for networks by the role URI.

% The used-on elements are a list of network types that are permitted to use the component.
% For example, if the network lookup returns a \texttt{PresentationNetwork} instance, 
% then the roleType XML element must contain "presentationLink" in its list of used-on elements.

% This concludes the section on the general implementation of Brel.
% Apart from networks, this section covered every part of XBRL and how Brel parses it.
% The next section will discuss network parsing in detail.

% In many places, the Brel API relies on composition over inheritance, which is a common design pattern in object oriented programming.
% It refers to a "has-a" relationship between classes, where one class points to another class using a field.
% For example, a \texttt{PresentationNetworkNode} has a \texttt{IReportElement}, 
% which can be a \texttt{Concept}-, \texttt{Abstract}- or any other report element instance.
% But \texttt{PresentationNetworkNode}s are not the only instances that have \texttt{IReportElement}s.
% The \texttt{IReportElement} instances are shared among networks as well as \texttt{ICharacteristic}s.
