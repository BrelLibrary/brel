\section{General implementation}
\label{sec:implementation_general}

Brel parses XBRL reports using an eager bottom-up approach
It starts with the smallest building blocks of XBRL reports - report elements.
After all report elements have been parsed, Brel moves on to parsing facts and their associated characteristics.
Next, Brel parses all networks and their associated resources.
Finally, Brel parses the components of the report.

Brel chooses this bottom-up approach because both networks and facts depend on report elements.
Networks depend on report elements since their nodes can point to report elements.
Facts depend on report elements since their characteristics can refer to concepts, dimensions and members.
Networks and facts often refer to the same report elements.
Therefore, their python classes should share the same report element instances.
The bottom-up approach ensures that all report elements are parsed before they are used by networks and facts.
In the next four sections, this chapter will briefly cover the four stages of Brel's bottom-up approach.

\subsection{Parsing report elements}
\label{sec:implementation_report_elements}

Report elements are the smallest building blocks of XBRL reports.
Therefore, they do not rely on any other XBRL elements and can be parsed first.
Report elements are defined in the taxonomy set of the XBRL report, which is a collection of \texttt{.xsd} files in the XML format.
These files are all stored locally on the user's computer.
Even though XBRL does not require the taxonomy set to be stored locally, Brel does.
However, Brel automatically downloads the taxonomy set from the internet if it is not already stored locally.
The mechanism for downloading the taxonomy set is discussed in section \ref{sec:implementation_dts_caching}.

Taxonomies contain three types of elements - linkbases, roles and report elements.
Linkbases are discussed in section \ref{sec:implementation_networks}.
Roles are discussed in section \ref{sec:implementation_components}.
Report elements are discussed in this section.

In XBRL, a taxonomy can refer to other taxonomies and associate them with a namespace prefix.
% In most instances, different taxonomies agree on which namespace prefix and URI a given taxonomy should use.
For now, the reader can assume that different taxonomies agree on which namespace prefix and URI a given taxonomy should use.
Brel ensures that this assumption holds true in a process called namespace normalization, which is discussed in section \ref{sec:implementation_namespace_normalization}.
If all taxonomies agree on a prefix and URI for a given taxonomy, 
then all report elements defined in that taxonomy inherit the same prefix and URI as part of their QName.

Report elements within a taxonomy are organized in a flat list of XML elements
Each XML element has a unique name attribute, which represents the local name of the report element's QName.
Since the Brel API lists six different types of report elements, Brel needs to decide which type of report element each XML element represents.
There is no single attribute in the XML element that indicates the type of report element.
Instead, Brel uses a combination of different attributes to determine the type of report element.
The process of determining the type of report element is outlined in the following table:

\begin{table}[H]
    \centering
    \begin{tabular}{|l|l|l|l|l|}
        \hline
        \makecell[l]{\textbf{Report element} \\ \textbf{type}}  & 
        % \textbf{priority} & 
        % \textbf{XML abstract attribute} &
        \makecell[l]{\textbf{XML abstract} \\ \textbf{attribute}} &
        \makecell[l]{\textbf{XML substitutionGroup} \\ \textbf{attribute}} & 
        % \textbf{XML type attribute} \\ \hline
        \makecell[l]{\textbf{XML type} \\ \textbf{attribute}} \\ \hline
        Concept     & "false"   &                                      &                            \\ \hline
        Hypercube   & "true"    & "xbrldt:hypercubeItem"                &                            \\ \hline
        Dimension   & "true"    & "xbrldt:dimensionItem"                &                            \\ \hline
        Member      & "true"    & "xbrli:item"                          & "domainItemType" \\ \hline
        Abstract    & "true"    & "xbrli:item"                          &                            \\ \hline
    \end{tabular}
    \caption{Determining the type of report element}
    \label{tab:determining_report_element_type}
\end{table}

Brel uses the table \ref{tab:determining_report_element_type} above to determine the type of report element.
It traverses the table from top to bottom and selects the first row where all conditions are met.
If a cell in the table is empty, then Brel ignores that condition. 

The table above does not contain a row for the type "LineItems".
The reason for this is that line items and abstracts can not be distinguished by their XML attributes.
They can only be distinguished by their position within a definition network.
Therefore, Brel parses line items and abstracts as abstracts.
Brel later uses the definition network to determine which abstracts are line items, and fixes the type of these abstracts accordingly.
This process is discussed in section \ref{sec:implementation_networks}.

Once all report elements have been parsed, Brel creates a lookup table for report elements.
Given a QName, the lookup table returns the corresponding report element instance.
This lookup table is used throughout the rest of the parsing process.

\subsection{Parsing facts}

Brel parses facts directly after parsing report elements.
Facts are parsed before networks because footnote networks can point to facts.

Facts are exclusively defined in the instance document of the XBRL report.
The instance document is an XML which contains a flat list of facts, syntactic contexts and units, which are represented by XML elements.
It may also contain a list of footnotes, which are discussed in section \ref{sec:implementation_networks}.

\textbf{Fact} XML elements contain the value of the fact as well as references to the syntactic context and unit.
The tag of the XML element is the QName of the concept of the fact. 

\textbf{Syntactic context} XML elements describe a subset of the characteristics of a fact.
They are different from \texttt{Context}s as defined by the Brel API.
A \texttt{Context} in Brel contains all the characteristics of a fact, whereas a syntactic context only contains the period, entity and dimensions of a fact.
During parsing, Brel uses syntactic contexts as a starting point for creating \texttt{Context} instances.
It then adds the remaining characteristics - the concept and the unit - to the \texttt{Context} instance.

\textbf{Unit} XML elements define, as the name suggests, the unit of a fact.

The reason why XBRL separates facts, syntactic contexts and units into three different XML elements is to reduce redundancy.
Facts can share the same syntactic context and unit.

Brel parses all facts by finding all fact XML elements and resolving their references to syntactic contexts and units.
It re-uses units, entities and dimensions across different facts. 

\subsection{Parsing Components}
\label{sec:implementation_components}

Components are the last section of the XBRL report that Brel parses.
Before parsing components, Brel has already parsed all report elements, facts and networks.
So far, this chapter has not discussed networks.
Since parsing networks is complex, it is discussed in its own section - section \ref{sec:implementation_networks}.
For now, the reader can assume that Brel has already parsed all networks, and that there is a lookup table for networks.

Components, like report elements, are defined in the taxonomy set of the XBRL report.
XBRL chooses to call them "roleTypes" instead of "components".
To parse all components, Brel scans all taxonomy files for roleType XML elements.
The roleType XML elements contain a three elements - A role URI, an optional description and a list of used-on elements.
\texttt{Component}s in Brel directly read both the role URI and the description from the roleType XML element.
To get the networks that are associated with a component, Brel indexes the lookup table for networks by the role URI.

The used-on elements are a list of network types that are permitted to use the component.
For example, if the network lookup returns a \texttt{PresentationNetwork} instance, 
then the roleType XML element must contain "presentationLink" in its list of used-on elements.

This concludes the section on the general implementation of Brel.
Apart from networks, this section covered every part of XBRL and how Brel parses it.
The next section will discuss network parsing in detail.

% In many places, the Brel API relies on composition over inheritance, which is a common design pattern in object oriented programming.
% It refers to a "has-a" relationship between classes, where one class points to another class using a field.
% For example, a \texttt{PresentationNetworkNode} has a \texttt{IReportElement}, 
% which can be a \texttt{Concept}-, \texttt{Abstract}- or any other report element instance.
% But \texttt{PresentationNetworkNode}s are not the only instances that have \texttt{IReportElement}s.
% The \texttt{IReportElement} instances are shared among networks as well as \texttt{ICharacteristic}s.
