\section{Implementation of Facts}

\subsection{Overview}

Facts in Brel function very similar to facts in XBRL. 
However, there are a couple of key differences between XBRL and Brel that are worth highlighting.
Before we dive into the details of facts in Brel, let us first look at how facts are represented in XBRL.

\subsection{Facts in XBRL}

In the context of XBRL, facts are represented as XML elements in an XBRL instance document. 
The following snippet shows an example of a fact in XBRL:

\begin{figure}[H]
    \caption{Example of a fact in XBRL}
    \label{fig:example_fact_xbrl}
    \begin{lstlisting}[language=XML]
        <us-gaap:ExtinguishmentOfDebtAmount 
          contextRef="c-216" 
          decimals="-6" 
          id="f-727" 
          unitRef="usd">
            121000000
        </us-gaap:ExtinguishmentOfDebtAmount>
    \end{lstlisting}\footnote[1]{taken from COCA COLA CO's 2023 Q2 report}
\end{figure}


Thinking back to how we defined facts in chapter TODO,
we can see that the above example does not contain all the information that we defined as being necessary for a fact.
In particular, the above example does only contain information about the concept, the value, and the unit of the fact.
Both the concept and the unit are merely links to other elements in the XBRL taxonomy, but do not contain any information about the concept or the unit themselves.
Furthermore, instead of providing information about the entity and the period, the example only contains a reference to a context element. 

\subsection{Contexts in XBRL}

In XBRL, contexts are used to provide information about the entity and the period of a fact. 
If the fact is part of a hypercube, the context also contains information about the dimensions and members of the fact.
Contexts are defined in the instance document using the \texttt{context} element.

Going back to our example from above, the context element referenced by the fact is defined as follows:

\begin{figure}[H]
    \caption{Example of the context referenced by the fact in figure \ref{fig:example_fact_xbrl}}
    \label{fig:example_context_xbrl}
    \begin{lstlisting}[language=XML]
        <context id="c-216">
            <entity>
                <identifier scheme="http://www.sec.gov/CIK">
                    0000021344
                </identifier>
                <segment>
                    <xbrldi:explicitMember dimension="dei:LegalEntityAxis">
                        ko:BottlingOperationsInAfricaMember
                    </xbrldi:explicitMember>
                </segment>
            </entity>
            <period>
                <startDate>2023-01-01</startDate>
                <endDate>2023-06-30</endDate>
            </period>
        </context>
    \end{lstlisting}
    \footnote[2]{taken from COCA COLA CO's 2023 Q2 report}
\end{figure}

The XML snippet fills in some of the missing information from the fact, namely:

\begin{itemize}
    \item The \textbf{entity} refers to \textit{who} is reporting. 
    The entity XML element provides us with a identifier for the entity, as well as information about where the identifier comes from.
    In this case, the identifier is a Central Index Key (CIK) provided by the US Securities and Exchange Commission (SEC).
    If we look up the identifier "0000021344" in the SEC's database, we can see that it corresponds to the Coca Cola Company.
    \item The \textbf{period} refers to \textit{when} the information is being reported.
    As described in chapter TODO, a period can either be a point in time or a duration.
    In this case, the period is defined as a duration, starting on January 1, 2023 and ending on June 30, 2023.
    \item The \textbf{dimensions} refer to additional information about the fact.
    This information is only relevant if the fact is part of a hypercube.
    Dimensions are defined as child elements of the \texttt{segment} element, 
    which is a child element of the \texttt{entity} element.
\end{itemize}

A observant reader might have noticed that the placement of dimensions in the context element is not consistent with the placement of both the entity and the period information of a fact.
The main reason for this inconsistency is that the XBRL specification was not designed with hypercubes in mind.
Hypercubes were added to the XBRL specification at a later point in time.

A second oddity of facts in XBRL is that they are not self-contained. 
Their definition is spread across multiple elements in the instance document.
Most noticably, the context element is defined separately from the fact element.
Since different XBRL facts reference the same context element, the incentive behind this design decision is clear: Reducing redundancy.
A XBRL instance document can contain thousands of facts, but only a handful of context elements.

However, this design falls apart as soon as we consider the case of hypercubes.
To illustrate the problem, consider the following numbers:

In their 2023 Q2 report, the Coca Cola Company 

\begin{enumerate}
    \item reported 1658 facts.
    \item defined 13 context elements without dimensions.
    \item defined 397 context elements with dimensions.
    \item used the same entity for all facts.
    \item uesd around 15 different periods for all facts.
\end{enumerate}

So the vast majority of all contexts exist to provide information about the dimensions of a fact.
However, the entity and the period information are mostly redundant.

\subsection{Units in XBRL}

In the same way that contexts are used to provide information about the entity and the period of a fact, units are used to provide information about the unit of a fact.
Units are also defined in the instance document using the \texttt{unit} element.
Going back to our example from above, the unit element referenced by the fact is defined as follows:

\begin{lstlisting}[language=XML]
    <unit id="usd">
        <measure>iso4217:USD</measure>
    </unit>
\end{lstlisting}\footnote[3]{taken from COCA COLA CO's 2023 Q2 report}

This XML snippet connects our intuitive understanding of the unit "usd" with a formal definition of the unit.
It does so by referencing the official ISO 4217 currency code for the US dollar.

\subsection{Concepts in XBRL}

As mentioned in chapter \ref*{chapter:xbrl}, concepts are the fundamental building blocks of XBRL.
Each fact is associated with exactly one concept.
Unlike units, periods, and contexts, concepts are not defined in the instance document.
Instead, they are defined in the XBRL taxonomy, which is a collection of XML schema documents.
In case of the above example, the concept is defined in the US GAAP taxonomy, which is a collection of XBRL taxonomies for US Generally Accepted Accounting Principles (GAAP).
This already gives us a hint about where to look for the definition of the concept.

The following snippet shows the definition of the concept in the US GAAP taxonomy:

\begin{lstlisting}[language=XML]
    <xs:element id="us-gaap_ExtinguishmentOfDebtAmount" name="ExtinguishmentOfDebtAmount" nillable="true" substitutionGroup="xbrli:item" type="xbrli:monetaryItemType" xbrli:balance="debit" xbrli:periodType="duration"/>
\end{lstlisting}\footnote[4]{taken from the US GAAP taxonomy}

The main purpose of this snippet is to provide a formal definition of the concept and to constrain the values that are allowed for the concept.
In particular, it tells us that the concept is a monetary item, that it has a debit balance, and that it has a duration period type.

\subsection{Facts in Brel}

In Brel, facts are represented as Python objects. 
When processing facts in XBRL, the information about the fact is spread across multiple elements in multiple documents.
The main goal of Brel is to provide a more intuitive interface for working with facts.
Whereas XBRL facts in XML, concepts, units, periods, etc. are all treated differently, Brel facts treat all aspects of a fact equally.