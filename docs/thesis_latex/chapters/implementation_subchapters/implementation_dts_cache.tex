\section{DTS Cache}
\label{sec:implementation_dts_cache}

The second key assumption that section \ref{sec:implementation_general} makes is that 
every file of both the taxonomy set and the XBRL report is stored locally on the user's computer.
In reality, this assumption does not hold true.
In the most common case, only the XBRL report is stored locally on the user's computer.
The taxonomy file within the XBRL report points to other taxonomy files, which are not stored locally.
As mentioned in section \ref{sec:concepts}, taxonomies may refer to other taxonomies.
Brel needs to resolve and download the transitive closure of all taxonomy references in order to parse the XBRL report.
Brel refers to this process as DTS caching.

\subsection{Discovery}

The letter D in DTS caching stands for "discoverable".
It suggests that Brel first needs to discover all taxonomy files that the XBRL report refers to.
Brel's approach is to parse the taxonomy file within the XBRL report and extract all taxonomy references from it.
% Brel then fetches all taxonomy files that are referenced by the taxonomy file within the XBRL report.
% It repeats this process until it has fetched all taxonomy files that the XBRL report refers to.
A taxonomy file might refer to other taxonomy files different ways.

\begin{itemize}
    \item \textbf{schemaRef} - The most common way is to refer to other taxonomy files using the \texttt{schemaRef} element.
    The \texttt{schemaRef} XML element contains a \texttt{href} attribute, which contains a URL that points to another taxonomy file.
    \item \textbf{linkbaseRef} - A taxonomy file may also refer to other taxonomy files using the \texttt{linkbaseRef} element.
    Like the \texttt{schemaRef} element, the \texttt{linkbaseRef} element contains a \texttt{href} attribute.
    \item \textbf{import} - A taxonomy file may also refer to other taxonomy files using the \texttt{import} element.
    The \texttt{import} element contains a \texttt{schemaLocation} attribute containing an URI that points to another taxonomy file.
    \item \textbf{include} - A taxonomy file may also refer to other taxonomy files using the \texttt{include} element.
    Similar to the \texttt{import} element, the \texttt{include} element contains a \texttt{schemaLocation} attribute containing an URI that points to another taxonomy file.
\end{itemize}

Whenever Brel parses a taxonomy file, it extracts all taxonomy references from it and adds them to a working set of taxonomy references.
Once Brel has parsed the current taxonomy file, it fetches the first taxonomy reference from the working set.
Brel then repeats the process of parsing the taxonomy file and extracting all taxonomy references from it.
The reader can think of this process as a breadth-first search of the taxonomy reference graph.
If Brel has already parsed a taxonomy file, then it does not need to parse it again.

\subsection{Downloading taxonomies}

Given a URI, downloading the taxonomy file that the URI points to is trivial for most URIs.
However, some URIs are relative URIs, 
which means that the URI indicates the location of the other taxonomy file relative to the current taxonomy file\cite{w3_relative_uri}.
Brel infers the domain name of relative URIs by remembering the domain name of the taxonomy file that the relative URI was referenced from.
It concatenates the domain name of the taxonomy file with the relative URI to form an absolute URI.

Relative URIs introduce another problem - File names for taxonomy files.
Since Brel downloads taxonomy files from the internet, it needs to store them locally on the user's computer.
Brel chooses to store taxonomy files in a directory called \texttt{dts\_cache} without using any subdirectories.
This means that Brel needs to ensure that all taxonomy files have unique file names.
The natural choice for file names is the local name of the taxonomy file's URI, which is not necessarily unique.
The next option is to use the URI itself as the file name. 
Since URIs may be relative, there might be multiple URIs that point to the same taxonomy file.
Additionally, tend to be very long and contain characters that are not allowed in file names.
The approach that Brel chooses is to generate a unique file name based on the absolute URI of the taxonomy file.
Brel strips the URI of all illegal characters to form a valid file name.

Using both the discovery and download mechanism, Brel is able to download all taxonomy files that the XBRL report refers to.
Brel stores the taxonomy files in the \texttt{dts\_cache} directory, 
which resolves the second key assumption that section \ref{sec:implementation_general} makes.
