% \section{DTS Cache}
% \label{sec:implementation_dts_cache}

% The second key assumption that section \ref{sec:implementation_general} makes is that 
% every file of both the taxonomy set and the XBRL report is stored locally on the user's computer.
% In reality, this assumption does not hold true.
% In the most common case, only the XBRL report is stored locally on the user's computer.
% The taxonomy file within the XBRL report points to other taxonomy files, which are not stored locally.
% As mentioned in section \ref{sec:concepts}, taxonomies may refer to other taxonomies.
% Brel needs to resolve and download the transitive closure of all taxonomy references in order to parse the XBRL report.
% Brel refers to this process as DTS caching.
\section{Discoverable Taxonomy Set (DTS) Caching}
\label{sec:implementation_dts_cache}

The second significant assumption made in section \ref{sec:implementation_general} is that all files pertaining to both the taxonomy set and the XBRL report are available locally on the user's computer. 
However, this is often not the case. 
Typically, only the XBRL report itself is stored locally. 
The taxonomy files referenced within the XBRL report usually point to additional taxonomy files that are not locally stored. 
As discussed in section \ref{sec:concepts}, taxonomies may include references to other taxonomies. 
To successfully parse the XBRL report, Brel must identify and download the complete set of all linked taxonomy references, a process known as DTS caching.

% \subsection{Discovery}

% The letter D in DTS caching stands for "discoverable".
% It suggests that Brel first needs to discover all taxonomy files that the XBRL report refers to.
% Brel's approach is to parse the taxonomy file within the XBRL report and extract all taxonomy references from it.
% % Brel then fetches all taxonomy files that are referenced by the taxonomy file within the XBRL report.
% % It repeats this process until it has fetched all taxonomy files that the XBRL report refers to.
% A taxonomy file might refer to other taxonomy files different ways.

% \begin{itemize}
%     \item \textbf{schemaRef} - The most common way is to refer to other taxonomy files using the \texttt{schemaRef} element.
%     The \texttt{schemaRef} XML element contains a \texttt{href} attribute, which contains a URL that points to another taxonomy file.
%     \item \textbf{linkbaseRef} - A taxonomy file may also refer to other taxonomy files using the \texttt{linkbaseRef} element.
%     Like the \texttt{schemaRef} element, the \texttt{linkbaseRef} element contains a \texttt{href} attribute.
%     \item \textbf{import} - A taxonomy file may also refer to other taxonomy files using the \texttt{import} element.
%     The \texttt{import} element contains a \texttt{schemaLocation} attribute containing an URI that points to another taxonomy file.
%     \item \textbf{include} - A taxonomy file may also refer to other taxonomy files using the \texttt{include} element.
%     Similar to the \texttt{import} element, the \texttt{include} element contains a \texttt{schemaLocation} attribute containing an URI that points to another taxonomy file.
% \end{itemize}

% Whenever Brel parses a taxonomy file, it extracts all taxonomy references from it and adds them to a working set of taxonomy references.
% Once Brel has parsed the current taxonomy file, it fetches the first taxonomy reference from the working set.
% Brel then repeats the process of parsing the taxonomy file and extracting all taxonomy references from it.
% The reader can think of this process as a breadth-first search of the taxonomy reference graph.
% If Brel has already parsed a taxonomy file, then it does not need to parse it again.

\subsection{Discovery Process in DTS Caching}

The 'D' in DTS caching represents "discoverable," implying that Brel's initial step is to identify all taxonomy files referenced by the XBRL report. 
Brel commences this process by parsing the taxonomy file included in the XBRL report and extracting all its taxonomy references. 
Taxonomy files may reference other taxonomy files in several ways:

\begin{itemize}
\item \textbf{schemaRef} - The most prevalent method is through the \texttt{schemaRef} element. 
This element contains a \texttt{href} attribute with a URL pointing to another taxonomy file.
\item \textbf{linkbaseRef} - Another way is using the \texttt{linkbaseRef} element. 
Similar to \texttt{schemaRef}, this element also includes a \texttt{href} attribute.
\item \textbf{import} - Taxonomy files might reference others using the \texttt{import} element, 
which has a \texttt{schemaLocation} attribute specifying a URI leading to another taxonomy file.
\item \textbf{include} - Similarly, the \texttt{include} element, 
containing a \texttt{schemaLocation} attribute, can also reference additional taxonomy files.
\end{itemize}

When Brel parses a taxonomy file, it identifies all the taxonomy references within and adds them to a list of references to be processed. 
After parsing a given taxonomy file, Brel selects the first reference from this list and repeats the process of parsing and extracting references. 
This approach resembles a breadth-first search through the taxonomy reference graph. 
If Brel has already processed a particular taxonomy file, it does not parse it again.

% \subsection{Downloading taxonomies}

% Given a URI, downloading the taxonomy file that the URI points to is trivial for most URIs.
% However, some URIs are relative URIs, 
% which means that the URI indicates the location of the other taxonomy file relative to the current taxonomy file\cite{w3_relative_uri}.
% Brel infers the domain name of relative URIs by remembering the domain name of the taxonomy file that the relative URI was referenced from.
% It concatenates the domain name of the taxonomy file with the relative URI to form an absolute URI.

% Relative URIs introduce another problem - File names for taxonomy files.
% Since Brel downloads taxonomy files from the internet, it needs to store them locally on the user's computer.
% Brel chooses to store taxonomy files in a directory called \texttt{dts\_cache} without using any subdirectories.
% This means that Brel needs to ensure that all taxonomy files have unique file names.
% The natural choice for file names is the local name of the taxonomy file's URI, which is not necessarily unique.
% The next option is to use the URI itself as the file name. 
% Since URIs may be relative, there might be multiple URIs that point to the same taxonomy file.
% Additionally, tend to be very long and contain characters that are not allowed in file names.
% The approach that Brel chooses is to generate a unique file name based on the absolute URI of the taxonomy file.
% Brel strips the URI of all illegal characters to form a valid file name.

% Using both the discovery and download mechanism, Brel is able to download all taxonomy files that the XBRL report refers to.
% Brel stores the taxonomy files in the \texttt{dts\_cache} directory, 
% which resolves the second key assumption that section \ref{sec:implementation_general} makes.

\subsection{Downloading Taxonomies}

Retrieving the taxonomy file from a given URI is straightforward for most URIs.
However, certain URIs are relative,
meaning the URI specifies the location of another taxonomy file in relation to the current one\cite{w3_relative_uri}.
Brel deduces the domain of relative URIs by recalling the domain from which the relative URI was referenced.
It then merges the domain of the current taxonomy file with the relative URI to create a complete URI.
% Relative URIs bring up an additional issue - the naming of taxonomy files.
% As Brel downloads these files from the internet, it must save them locally.
% Brel opts to store taxonomy files in a folder named \texttt{dts\_cache}, without subfolders.
% Hence, Brel needs to ensure each taxonomy file has a distinct name.
% The initial option for naming is to use the local name from the taxonomy file's URI, but this may not be unique.
% Another possibility is to utilize the URI itself as the file name.
% However, since URIs can be relative and typically long, including characters unsuitable for file names, this is impractical.
% Brel's chosen solution is to create a unique file name derived from the complete URI of the taxonomy file,
% removing any characters that are invalid for file names.
When storing taxonomy files, Brel names them based on their absolute URIs, ensuring each file has a unique name.
Since URI-based file names are not always valid, Brel removes any illegal characters to form a valid file name.

By employing the discovery and downloading mechanisms,
Brel successfully retrieves all taxonomy files referenced by the XBRL report.
Brel then saves these files in the \texttt{dts\_cache} directory,
addressing the second crucial assumption outlined in section \ref{sec:implementation_general}.
