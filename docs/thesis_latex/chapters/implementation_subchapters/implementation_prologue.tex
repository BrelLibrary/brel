% \section{Overview}

% The creation of brel could be divided into three main parts 
% The design of the API, the implementation of the API and testing of the API.
% The bulk of both the API design and testing will be discussed in chapter \ref{chapter:results}.
% This chapter will focus on the implementation of the API.

% The implementation of the API was done in the programming language Python.
% We chose python because it is a high level language and one of the most popular programming languages in the world.
% Python is also quite popular outside of the computer science and software engineering communities as well.
% Its simplicity and readability makes it a good choice for beginners and non-programmers.

% Brel parses XBRL reports, which are used by companies to report their financial information.
% So the target audience of brel are accountants and financial analysts.
% These people are not necessarily programmers, 
% so using a language that is easy to read and understand is just as creating an API that is easy to read and understand.

% Chapter \ref{chapter:xbrl} already discussed the structure of XBRL reports. 
% In summary, XBRL reports are XML documents that contain the following information:

% \begin{itemize}
%     \item A flat list of facts
%     \item A list of report elements, many of which are created by external taxonomies
%     \item A system for identifying different report elements, and other objects in the report.
%     Brel uses QNames.
%     \item Components which form the chapters of the report.
%     \item Networks which link report elements together to model relationships between them.
% \end{itemize}

% This chapter will cover all the aforementioned elements of XBRL reports in their own section.
% It will discuss the difficulties of parsing these elements and how brel solves these difficulties.
% The first section is about QNames, which are used extensively throughout XBRL reports.

The implementation of the API closely follows the design of the API introduced in chapter \ref{chapter:api}.
% The implementation of the API is done in the programming language Python.
Brel is implemented in the programming language Python.
Python is a high level language and one of the most popular programming languages in the world.
Python is also popular outside of the computer science and software engineering communities as well.
According to the 2020 Stack Overflow Developer Survey, Python is the fourth most popular programming language among all respondents, 
not just professional developers \cite{stack_overflow_2020}.

The reader may view the implementation of Brel as a translation from XBRL reports to python objects.
Most of the translations from the design to the implementation are straightforward and will only be mentioned briefly, 
which the first section of this chapter will cover.

There are three key areas where the Brel's implementation differs from the underlying XBRL standard - DTS caching, namespace normalization and networks.
Each area has its own dedicated section in this chapter.

This chapter exclusively focuses on XBRL reports in XML format.
