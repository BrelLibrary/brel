\section{Abstract}
% \addcontentsline{toc}{chapter}{Preface}
%\addcontentsline{toc}{chapter}{Zusammenfassung}

% TODO: Write the abstract

% XBRL is a standard for financial reporting that is both human and machine-readable.
% Regulatory bodies, such as the United States Securities and Exchange Commission (SEC), 
% require companies to submit their financial reports in XBRL format.
% Despite the standard's widespread adoption, there is a lack of tools for working with XBRL reports.
% Furthermore, the standard recently underwent a major update,
% introducing a format-agnostic approach to XBRL reporting called the Open Information Model (OIM).

% This thesis presents Brel, a Python library for working with XBRL reports.
% It provides a high-level interface for extracting and querying data from XBRL reports.
% The thesis outlines both the design and implementation of Brel.
% Brel is designed to be easy to use and to abstract away the complexities of the XBRL standard.
% Whilst Brel is designed to work with the OIM and the XML-based XBRL format,
% it is also designed to be extensible to support future XBRL formats.
% The library is evaluated based on its correctness, robustness, performance, usability,
% and how it measures up to comparable tools.
% The evaluation shows that Brel is a robust and accurate tool for working with XBRL reports.

% XBRL, a standard for financial reporting, is readable by both humans and machines.
XBRL is a standard for financial reporting that is both human and machine-readable.
The United States Securities and Exchange Commission (SEC), among other regulatory bodies,  
% mandates the submission of financial reports in XBRL format.
mandates all public companies to file their financial reports in XBRL format.
Despite the widespread use of XBRL, there is a shortage of tools for handling XBRL reports,
especially in the area of open-source software.
Additionally, the standard has been recently updated,  
introducing the format-independent Open Information Model (OIM), 
which acts as a logical, syntax-agnostic model for XBRL reporting.
% which acts as a logical, syntax-agnostic layer for XBRL reporting.

This thesis introduces Brel, a Python library for XBRL report processing.  
It offers an API for data extraction from XBRL reports.
% The thesis details Brel's design and development,
% with the first half focusing on the XBRL standard and the Brel API's design.
The first part of the thesis discusses the XBRL standard and the design of the Brel API.
Brel aims to simplify usage of the XBRL standard.
While Brel supports both OIM and XML-based XBRL formats,  
its design allows for future XBRL format adaptability.

% The second half of the thesis focuses on Brel's implementation and evaluation.
The latter half of the thesis delves into Brel's implementation and evaluation.
% The evaluation of the library focuses on its correctness, robustness, performance, user-friendliness,  
% and comparison with similar tools.
The library's evaluation emphasizes its correctness, robustness, performance, usability,
and its standing relative to a comparable tool.
% The evaluation demonstrates that Brel is a reliable and precise tool for working with XBRL reports.
% This assessment confirms Brel's reliability and precision in processing XBRL reports.
This assessment confirms Brel's reliability and precision in processing XBRL reports.
