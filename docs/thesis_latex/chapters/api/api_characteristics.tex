\section{Characteristics}
\label{sec:characteristics}

Characteristics are used to describe the position of a fact along a dimension.
Some of them rely on report elements for description, while others introduce new concepts. 
All characteristics share a common interface \texttt{ICharacteristic}.
Each characteristic acts as a aspect-value pair, 
where the aspect characterizes the dimension's axis and the value details the position of the fact along the axis.
The interplay between aspect and characteristics classes is illustrated in figure \ref{fig:characteristics}.

\begin{figure}[H]
    \centering
    \includegraphics[width=\textwidth]{images/brel_characteristics_classes.png}
    \caption{The interplay between aspects and characteristics}
    \label{fig:characteristics}
\end{figure}

The \texttt{ICharacteristic} interface is integral to the Brel API, which is why it is described first.
Next up is the Aspect class, followed by the different types of characteristics.

\subsection{ICharacteristic}

As I have pointed out in the previous section, characteristics are used to describe the position of a fact along a dimension using an aspect-value pair.
An aspect is a description of the dimension's axis, while the value details the position of the fact along the axis.

The \texttt{ICharacteristic} interface follows this definition directly by providing the methods \texttt{get\_aspect} and \texttt{get\_value}.

\subsection{Aspect}

Aspects are used to describe the axis of a dimension.
Each instance of the \texttt{Aspect} class represents a single aspect.
The core aspects - Concept, Entity, Period and Unit - are all instances of the \texttt{Aspect} class.
In addition, the core aspects are statically available as public constants of the \texttt{Aspect} class.
The fields in question are \texttt{Aspect.CONCEPT}, \texttt{Aspect.ENTITY}, \texttt{Aspect.PERIOD} and \texttt{Aspect.UNIT}.

The \texttt{Aspect} class provides a method for getting the name of the aspect, called \texttt{get\_name}.
The name of an aspect is a string instead of the QName class.
The main reason for this is that the core aspects are available globally without the use of namespaces.
Most facts exclusively use the core aspects.
So the namespaces of QNames would only add unnecessary clutter.
The only exception to this rule are dimensions, where the name of the dimension is a QName.
Still, QNames can just be emulated by using strings using their expanded name format.
\footnote{The expended name format of a QName is \texttt{namespace\_prefix:local\_name}.\cite{w3_expanded_names}}
\cite{w3_qnames}

Similar to the \texttt{IReportElement} interface, the \texttt{Aspect} class also provides a method for getting the label of the aspect.
The \texttt{get\_labels} method returns a list of labels, since an aspect can have multiple labels.

Finally, the \texttt{is\_core} method returns whether the aspect is a core aspect or not.

\subsection{Concept Characteristic}

The \texttt{ConceptCharacteristic} indicates which concept a fact uses.
From the perspective of hypercubes, the \texttt{Aspect.CONCEPT} aspect is a dimension of concepts 
and the actual \texttt{Concept} report element is a point along that dimension.
The dimension characteristic is the only characteristic that every context has to have.

The \texttt{ConceptCharacteristic} implements \texttt{ICharacteristic} as one would expect.
\texttt{get\_aspect} returns \texttt{Aspect.CONCEPT} and \texttt{get\_value} returns the concept that the characteristic describes.

\subsection{Entity Characteristic}

The \texttt{EntityCharacteristic} dictates which entity a fact belongs to.
From the perspective of hypercubes, the \texttt{Aspect.ENTITY} aspect is a dimension of entities
An entity is a legal entity, such as a company and
can be identified by a tag and a scheme\footnote{Schemes tend to be URLs.} that acts as the namespace of the tag.
% Entities are uniquely identified by an identifier and the scheme\footnote{Schemes tend to be URLs.} that defines the identifier.
Both of these values are combined into a single string using the notation \texttt{\{scheme\}tag}.\footnote{The notation is similar to the Clark notation for QNames.\cite{w3_qnames}}

The \texttt{EntityCharacteristic} implements the \texttt{ICharacteristic} interface,
where \texttt{get\_aspect} returns \texttt{Aspect.ENTITY} and \texttt{get\_value} gives the string representation of the entity as described above.

\subsection{Period Characteristic}

The \texttt{PeriodCharacteristic} describes the period of a fact.
Periods can be either instant or duration, which can be checked using the \texttt{is\_instant} method.

The methods \texttt{get\_start\_date} and \texttt{get\_end\_date} return the start and end date of the period respectively.
If the period is instant, the methods raise an exception, since instant periods do not have a start or end date.
Conversely, the method \texttt{get\_instant} returns the instant of the period.
If the period is duration, the method raises an exception, since duration periods do not have an instant.
All three methods return a date of type \texttt{datetime.date}, which is a standard Python class for representing dates.

Again, period characteristics implement the \texttt{ICharacteristic} interface.
\texttt{get\_aspect} returns \texttt{Aspect.PERIOD} and \texttt{get\_value} returns the period characteristic itself.
The reason why \texttt{get\_value} returns itself is that there is no basic type for representing XBRL periods in python.
The python \texttt{datetime} module, which is the de-facto standard for representing dates in python, 
does not provide a class for representing both instant and duration periods in a single class. 

\subsection{Unit Characteristic}

The \texttt{UnitCharacteristic} describes the unit of a fact.
Like all other characteristics before it, the \texttt{UnitCharacteristic} represents a point along the unit dimension 
and it implements the \texttt{ICharacteristic} interface.
From a semantic point of view, the unit characteristic also defines the type of the fact's value.
A fact with a unit of \texttt{USD} has a value of type \texttt{decimal} and a fact with a unit of \texttt{date} has a value representing a date.

Units come in one of two forms - simple and complex.
Simple units are atomic units, such as \texttt{USD} or \texttt{shares}.
Complex units are composed of multiple simple units, such as \texttt{USD per share}.
All complex units are formed by dividing one or more simple units by zero or more simple units.

\begin{figure}[H]
    \centering
    \caption{Schematic of composition of complex units}
    $$\frac{num\_unit_1 \cdot num\_unit_2 \cdot ...}{1 \cdot denom\_unit_1 \cdot denom\_unit_2 \cdot ...}$$
    \label{fig:complex_unit}
\end{figure}

Brel represents the complex unit in figure \ref{fig:complex_unit} using two lists of simple units.
The method \texttt{get\_numerators} returns the list of simple units in the numerator, \texttt{get\_denominators} returns the denominators.\footnote{The returned list of denominators does not contain the implicit denominator of 1.}

Similar to the \texttt{PeriodCharacteristic}, the \texttt{UnitCharacteristic} does not have a basic type for representing XBRL units.
Instead, it returns itself when \texttt{get\_value} is called.
The method \texttt{get\_aspect} returns \texttt{Aspect.UNIT} as expected.
% TODO: Make unit characteristic get_value return itself in code base

\subsection{Dimension Characteristics}

There are two categories of dimension characteristics in XBRL - typed and explicit.

Typed dimension characteristics are used to describe a custom axis, along which a fact is positioned.
The kind of values along this custom axis are of a specific type.
Like every other characteristic, typed dimension characteristics implement the \texttt{ICharacteristic} interface.

As we have seen in section \ref{sec:api_report_elements}, custom dimensions are represented as a \texttt{Dimension} report element in Brel.
So the aspect of a typed dimension characteristic should represent a \texttt{Dimension} report element.
Luckily, \texttt{Dimension} objects are essentially just a name, which is represented by a QName.
Therefore, the \texttt{get\_aspect} method of the \texttt{TypedDimensionCharacteristic} class returns the QName of the dimension as a string.

% Of course, the characteristic should also provide access to the \texttt{Dimension} object itself, 
% which is why the \texttt{get\_dimension} exists. 
The characteristic also provides direct access to the \texttt{Dimension} object itself via the \texttt{get\_dimension} method.
Think of \texttt{get\_dimension} as a more complete version of \texttt{get\_aspect}.

As the name suggests, the value of a typed dimension characteristic is of a specific type.
The \texttt{get\_value} method should reflect this accordingly.
It should return the value in a type that encompasses all possible values of the dimension.
The most general type of any value in XBRL is a string.

The actual type of the value is determined by the \texttt{get\_type} method of the \texttt{Dimension} element.
\footnote{The \texttt{Dimension} object returned by \texttt{get\_dimension} is guaranteed to be a typed dimension with \texttt{is\_explicit} returning \texttt{False}.}
Naturally, Brel provides helper methods for converting the value into the type that most appropriately represents the value.
These helper methods are not part of the minimal API described in this chapter, but they are part of the full API.

Explicit dimensions are the second category of custom characteristic.
They are extremely similar to typed dimensions, but they do not have a type.
Instead of a type, they have a set of possible values.

The \texttt{ExplicitDimensionCharacteristic} class is almost identical to the \texttt{TypedDimensionCharacteristic} class.
The main difference between the two is that \texttt{get\_value} returns a \texttt{Member} object instead of a string.

