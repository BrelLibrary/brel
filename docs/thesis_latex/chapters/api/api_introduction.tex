This chapter describes the Brel API.
It serves as a top-down overview of what Brel is capable of.
The API differs from the underlying XBRL standard in key areas and aims to abstract and simplify XBRL,
while still providing access to the full power of XBRL when needed.

The chapter is not intended to be a complete reference.
Brel contains various helper functions and classes not described in this chapter.
A complete reference of the Brel API can be found in the Brel API documentation\cite{brel_api}.
% The classes and methods described in this chapter serve as the minimal set of functionality needed to fully access all of Brel's features.
The classes and methods described in this chapter define the minimal set of functionality needed to fully access all of Brel's features.
They are designed to cover the underlying XBRL standard in a way that is easy to use and understand.
% All of the helper functions and classes in Brel can be implemented using the API described in this chapter.
Every additional feature of Brel could be implemented using the API described in this chapter.
\footnote{Obviously, some of the helpers directly access the underlying XBRL standard and are not implemented using the API to avoid unnecessary overhead.}
The API of Brel can be divided into different parts, all of which are described in this chapter.

\begin{enumerate}
    \item \textbf{Core} - The first part describes the \texttt{Core} of Brel, which consists of Filings, Facts, Components and QNames.
    \item \textbf{Characteristics} - The second part describes the \texttt{Characteristics} of Brel, which covers Concepts, Entities, Periods, Units and Dimensions.
    \item \textbf{Report Elements} - The third part covers \texttt{Report Elements}, specifically Concepts, Members, Dimensions, Line Items, Hypercubes and Abstracts.
    \item \textbf{Resources} - The fourth part covers \texttt{Resources}, specifically Labels, References and Footnotes.
    \item \textbf{Networks} - Finally, the \texttt{Networks} part describes how networks and their nodes are represented in Brel.
\end{enumerate}

All of these parts should sound very familiar, since they were extensively discussed in chapter \ref{chapter:xbrl}.
Even though the previous chapter was already light on XML implementation details, 
the API described completely abstracts away the underlying XML structure.
\footnote{The only exception is the \texttt{QName} class, which is an almost direct representation of the XML QName type.}
This chapter will answer research questions \ref{itm:research_question_1} and \ref{itm:research_question_2}.

Again, the current implementation of Brel is not complete.
It does not allow for the creation of new filings, facts, components, etc.
Its main purpose is to provide a way to access and analyze existing filings.
Brel also does not analyze the semantics of the underlying reports.
