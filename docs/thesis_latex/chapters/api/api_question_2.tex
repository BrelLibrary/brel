\section{Answering research question 2}
\label{sec:answer_research_question_2}

The second half of this chapter introduced Brel's representation of networks and resources, 
both of which are not yet covered by the OIM.
Therefore, a lot of the design decisions in this half of the chapter are not based on the OIM.

Research question \ref{itm:research_question_2} consists of two parts.

\begin{itemize}
    \item First, the question asks how the non-OIM parts of XBRL can be converted into a python API.
    \item Second, it asks how this API can be made consistent with the OIM.
\end{itemize}

The first part of the question is answered by the second half of this chapter in a constructive fashion.
This chapter an API for networks and resources, which break down the non-OIM parts of XBRL into their core components.

Answering the second part of the question requires a more abstract approach. 
In short terms, Brel marries the OIM and the non-OIM parts of XBRL by introducing report elements and characteristics.

The first half of this chapter already introduced a python API that is based on the OIM.
The OIM's main goal was to report facts.
Each fact has characteristics such as concepts, explicit dimensions, entities, etc.

The second half of this chapter was mostly concerned with networks and resources.
Networks point to report elements among other things.

However, report elements were already introduced as part of the OIM, 
even though they are not strictly necessary for reporting facts.
In fact, apart from concepts, dimensions and members, the OIM never even mentions any report elements.

The bridge between the two halves are characteristics and report elements, more specifically, 
there are three types of characteristics that are just wrappers around report elements.
These characteristics are the concept characteristic, the explicit dimension characteristic and the typed dimension characteristic.
The interplay between characteristics and report elements is illustrated in figure \ref{fig:brel_combine_api}.

\begin{figure}[H]
    \centering
    \includegraphics[width=0.8\textwidth]{images/brel_combine_api.png}
    \caption{The interplay between characteristics and report elements}
    \label{fig:brel_combine_api}
\end{figure}

Where the OIM uses the characteristics to describe facts, networks use the report elements inside the characteristics.
Of course, the class \texttt{Concept} for example could have just implemented both the \texttt{IReportElement} and the \texttt{ICharacteristic} interface.
However, concepts in facts and concepts in networks are used in different contexts, and they should not be used interchangeably.
Therefore, Brel uses two different classes for the two different use cases.
% However, Brel's API took a detour from the OIM when it introduced report elements.
% % Where the Brel API differs from the OIM so far is in its introduction of report elements.
% Yes, the OIM introduces concepts, dimensions and members, but it does not categorize them under a common umbrella term
% \footnote{The OIM would technically group concepts and members under the term "dimension", 
% but it overloads the term "dimension" so many times that it is not clear what it refers to in any given context.
% This grouping is also different from report elements semantics-wise.}.
% In the OIM, these three terms describe three completely unrelated things, and they are unrelated if one only considers the OIM.
% % Brel simply takes the liberty of grouping them under the term "report element".
% % However, the Brel API also aims to cover the parts of XBRL that are not yet covered by the OIM.
% % The non-OIM parts of XBRL are networks, components and resources.
% % Networks require elements like concepts, dimensions and members to be treated in a homogeneous way.
% % The exact reasoning behind this will be explained in the second half of this chapter.
% Brel simply takes the liberty of grouping them under the term "report element".
% It then renames the report elements by introducing the "characteristic" suffix.
% So instead of "concept", Brel uses "concept characteristic", which is a different class in the Brel API.
% % The way Brel bridges the gap between the OIM and the non-OIM parts of XBRL is by introducing characteristics and report elements.

% Characteristics are used for facts, while report elements are used for networks.
% This is the reason why Brel uses \texttt{ConceptCharacteristic} instead of \texttt{Concept} when talking about the concept characteristic of a fact.
% Sure, a \texttt{ConceptCharacteristic} is in essence a wrapper around a \texttt{Concept}, 
% but the two classes are used in different contexts.

This concludes the chapter on the Brel API.
It covered both the OIM and the non-OIM parts of XBRL.
It also explained how the two parts can be combined into a single API, in turn answering both research questions \ref{itm:research_question_1} and \ref{itm:research_question_2}.
With both the API and the underlying XBRL standard covered, the next chapter will introduce the implementation of the Brel API.
