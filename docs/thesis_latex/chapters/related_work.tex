\chapter{Related work}
\label{sec:related_work}

\section{XBRL}
As mentioned in section \ref{sec:introduction}, this thesis is based on the XBRL standard\cite{xbrl} originally created by Charles Hoffman.
The XBRL standard is a complex standard consisting of many different parts.
The parts of the XBRL standard that are relevant to this thesis are the Open Information Model (OIM)\cite{oim}, 
the XBRL 2.1 specification\cite{xbrl21}, 
the extension for dimensional reporting\cite{xbrl_dimensions}
and the specification for generic links\cite{xbrl_generic_links}.
The XBRL 2.1 specification and the OIM overlap in some areas, but the OIM is a partial rewrite of the XBRL 2.1 specification.
It does not contain all the features of the XBRL 2.1 specification, but it is a lot easier to understand.
The OIM also covers some aspects of XBRL dimensions, but does not contain any features from XBRL generic links.

\section{The XBRL Book}
To properly understand the XBRL specification, one already needs to have a good understanding of both XML and XBRL.
This makes it difficult for beginners to get started with XBRL.
To address this problem, Dr. Ghislain Fourny wrote "The XBRL Book"\cite{fourny2023xbrl}.
The book is a comprehensive guide to XBRL and covers all the important aspects of the XBRL standard, including the rather recent OIM.

\section{Arelle}
Arelle\cite{arelle} is an open source XBRL platform written in Python.
At the time of writing, Arelle is the most complete open source XBRL platform.
It supports all the features of the XBRL 2.1 specification and the OIM.
Like Brel, Arelle is also written in Python and is open source.
Unlike Brel, Arelle is a complete XBRL platform and not just a python library.

\section{Xule}
Xule\cite{xule} is a rule language for XBRL.
It is a declarative language that allows users to write rules that can be used to validate XBRL reports.
Xule is written in Python and is open source.
It is also part of the Arelle project and is used by Arelle to validate XBRL reports.
Think of Xule as a domain-specific language for XBRL validation rules.
Even though Xule is not part of this thesis, Brel should be able to support Xule in the future.
