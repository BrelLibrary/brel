\chapter{Related work}
\label{sec:related_work}

The goal of this thesis is to create a new open source XBRL library that is largely based on the OIM.
When it comes to work related to Brel, there are three main areas of interest.
First, there is the XBRL specification itself and interpretations of it.
Second, there are other XBRL libraries and platforms that are similar to Brel. This also includes public databases of XBRL reports.
Third, there are the requirements set by authorities and other organizations that XBRL processors have to fulfill.

\section{XBRL Specification}
As mentioned in section \ref{sec:introduction}, this thesis is based on the XBRL standard\cite{xbrl} originally created by Charles Hoffman.
The XBRL standard is a complex standard consisting of many different parts.
The parts of the XBRL standard that are relevant to this thesis are the Open Information Model (OIM)\cite{oim}, 
the XBRL 2.1 specification\cite{xbrl21}, 
the extension for dimensional reporting\cite{xbrl_dimensions}
and the specification for generic links\cite{xbrl_generic_links}.
The XBRL 2.1 specification and the OIM overlap in most areas, but the OIM is a partial rewrite of the XBRL 2.1 specification.
The OIM does not contain all the features of the XBRL 2.1 specification, but it is a lot easier to understand.
The OIM also covers some aspects of XBRL dimensions, but does not contain any features from XBRL generic links.

\section{The XBRL Book}
To properly understand the XBRL specification, one already needs to have a good understanding of both XML and XBRL.
This makes it difficult for beginners to get started with XBRL.
To address this problem, Dr. Ghislain Fourny wrote "The XBRL Book"\cite{fourny2023xbrl}.
The book is a comprehensive guide to XBRL and covers all the important aspects of the XBRL standard, including the rather recent OIM.

\section{Arelle}
Arelle\cite{arelle} is an open source XBRL platform written in Python.
At the time of writing, Arelle is the most complete open source XBRL platform.
It supports all the features of the XBRL 2.1 specification and the OIM.
Like Brel, Arelle is also written in Python and is open source.
Unlike Brel, Arelle is a complete XBRL platform and not just a python library.

\section{Xule}
Xule\cite{xule} is a rule language for XBRL.
It is a declarative language that allows users to write rules that can be used to validate XBRL reports.
Xule is written in Python and is open source.
It is also part of the Arelle project and is used by Arelle to validate XBRL reports.
Think of Xule as a domain-specific language for XBRL validation rules.
Even though Xule is not part of this thesis, Brel should be able to support Xule in the future.

\section{EDGAR}
\label{sec:edgar}
The SEC (US \texttt{S}ecurities and \texttt{E}xchange \texttt{C}ommission) maintains a system called EDGAR \cite{sec_edgar}. \footnote{not to be confused with the U.S. SEC, which stands for U.S. Soybean Export Council}
EDGAR is a system that companies can use to submit their financial reports to the SEC.
EDGAR stands for \textbf{E}lectronic \textbf{D}ata \textbf{G}athering, \textbf{A}nalysis, and \textbf{R}etrieval.
Filings submitted to EDGAR have to be in XBRL format.
The SEC provides public access to the filings submitted to EDGAR. \footnote{\url{https://www.sec.gov/edgar/search/}}

\section{ESEF filings}
The european counterpart to the SEC is called ESMA (\textbf{E}uropean \textbf{S}ecurities and \textbf{M}arkets \textbf{A}uthority).
It also maintains a database of ESEF filings\cite{esma_database}.
ESEF stands for \textbf{E}uropean \textbf{S}ingle \textbf{E}lectronic \textbf{F}ormat.
ESEF is a standard for XBRL reports in the EU.
Similar to EDGAR, the ESEF database is publicly accessible. \footnote{\url{https://filings.xbrl.org/}}
An interesting aspect of this database is that it is hosted by XBRL International, the organization that maintains the XBRL standard.

\section{SEC - Interactive Data Public Test Suite}
\label{sec:idpts}
To ensure that the XBRL processors that the companies use to create their XBRL reports are compliant with the XBRL standard, the SEC created the Interactive Data Public Test Suite\cite{sec_idpts}.
This test suite contains an extensive collection of XBRL reports that are used to test XBRL processors.
The SEC provides this test suite free of charge and it is available on their website.

Brel uses this test suite to test its XBRL processor.
At the time of writing, Brel does not pass all the tests in the test suite, since Brel does not support all the features of the XBRL standard.
However, building a processor that passes all the tests is not a feat that can feasibly be accomplished by a single person in the scope of a master's thesis.

\section{ESMA Conformance Suite}
The European Securities and Markets Authority (ESMA) also maintains a test suite for XBRL processors\cite{esma_conformance_suite}.
This test suite is similar to the SEC's Interactive Data Public Test Suite, but it is maintained by a different authority.
Another key difference is that the ESMA Conformance Suite facilitates automated testing of xHTML reports.

Brel does not support xHTML reports, so the ESMA Conformance Suite is not relevant for this thesis.
However, Brel should be able to support xHTML reports in the future.
