\chapter{Related work}
\label{sec:related_work}

% The goal of this thesis is to create a new open source XBRL library that is largely based on the OIM.
% When it comes to work related to Brel, there are three main areas of interest.
% First, there is the XBRL specification itself and interpretations of it.
% Second, there are other XBRL libraries and platforms that are similar to Brel. This also includes public databases of XBRL reports.
% Third, there are the requirements set by authorities and other organizations that XBRL processors have to fulfill.
The goal of this thesis is to create a new open-source XBRL library, primarily based on the OIM.
In the context of Brel, three main areas of interest exist.
Firstly, it involves considering the XBRL specification and its interpretations.
Secondly, it encompasses the examination of other XBRL libraries and platforms similar to Brel, including public databases of XBRL reports.
Lastly, it involves reviewing the requirements set by authorities and other organizations that XBRL processors must fulfill.

\section{XBRL Specification}
% As mentioned in section \ref{sec:introduction}, this thesis is based on the XBRL standard\cite{xbrl} originally created by Charles Hoffman.
% The XBRL standard is a complex standard consisting of many different parts.
% The parts of the XBRL standard that are relevant to this thesis are the Open Information Model (OIM)\cite{oim}, 
% the XBRL 2.1 specification\cite{xbrl21}, 
% the extension for dimensional reporting\cite{xbrl_dimensions}
% and the specification for generic links\cite{xbrl_generic_links}, 
% all of which will be covered in chapter \ref{chapter:xbrl}.
As indicated in Section \ref{sec:introduction}, this thesis builds upon the XBRL standard originally authored by Charles Hoffman\cite{xbrl}.
The XBRL standard comprises numerous components, and this thesis specifically focuses on the following components: 
the Open Information Model (OIM)\cite{oim}, 
the XBRL 2.1 specification\cite{xbrl21}, 
the extension for dimensional reporting\cite{xbrl_dimensions}, 
and the specification for generic links\cite{xbrl_generic_links}.
Chapter \ref{chapter:xbrl} will delve into these components in detail.
% Besides the XBRL standard, Charles Hoffman also maintains a personal website\cite{seattle_method}
% that contains a wealth of information about XBRL and good practices for working with it.
In addition to the XBRL standard, Charles Hoffman manages a personal website\cite{seattle_method},
which offers information about XBRL and good practices for its utilization.

% Whilst the XBRL 2.1 specification and the OIM overlap in numerous areas, the OIM does not encompass all the features of the XBRL 2.1 specification.
% The OIM does not contain all the features of the XBRL 2.1 specification, but it is a lot easier to understand.
% The OIM also covers some aspects of XBRL dimensions, but does not contain any features from XBRL generic links.
% Additionally, the OIM addresses certain aspects of XBRL dimensions but does not incorporate any features from XBRL generic links.

\section{The XBRL Book}
% To properly understand the XBRL specification, one already needs to have a good understanding of both XML and XBRL.
Understanding the XBRL specification requires a good grasp of both XML and XBRL.
% This makes it difficult for beginners to get started with XBRL.
% To address this problem, Dr. Ghislain Fourny wrote "The XBRL Book"\cite{fourny2023xbrl}.
% The book is a comprehensive guide for XBRL and covers all the important aspects of the XBRL standard, including the rather recent OIM.
To help newcomers, Dr. Ghislain Fourny has authored "The XBRL Book" \cite{fourny2023xbrl}, 
which serves as a comprehensive guide to XBRL. This book covers all important aspects of the XBRL standard, including the relatively recent OIM, 
making it an invaluable resource for those looking to learn about XBRL.

\section{Arelle}
Arelle\cite{arelle} stands as an open-source XBRL platform. 
As of the current writing, Arelle holds the distinction of being the most comprehensive open-source platform in its category. 
It provides support for all features found in the XBRL 2.1 specification and the OIM. 
Similar to Brel, Arelle is implemented in Python and is available as open-source software. 
However, it's important to note that Arelle is a complete XBRL platform, in contrast to Brel, which is primarily a Python library. 
% To put it simply, Arelle can be thought of as the "Excel for XBRL."
The reader can think of Arelle as the "Excel for XBRL."
% Arelle\cite{arelle} is an open source XBRL platform.
% At the time of writing, Arelle is the most fully open source platform of its kind.
% It supports all the features of the XBRL 2.1 specification and the OIM.
% Like Brel, Arelle is also written in Python and is open source.
% Unlike Brel, Arelle is a complete XBRL platform and not just a python library.
% Think of Arelle as "Excel for XBRL".

\section{Xule}
% Xule\cite{xule} is a rule language for XBRL.
% It is a declarative language that allows users to write rules for validating XBRL reports.
% It is part of the Arelle project and is used by Arelle to validate reports.
% Think of Xule as a domain-specific language for XBRL validation rules.
% Even though Xule is not part of this thesis, Brel should be able to support Xule in the future.
Xule\cite{xule} serves as a rule language tailored for XBRL. 
This declarative language empowers users to craft rules for the validation of XBRL reports. 
The Arelle project employs Xule to validate reports. 
Conceptually, Xule can be likened to a domain-specific language designed specifically for XBRL validation rules. 
While Xule is not directly incorporated into this thesis, it is worth noting that Brel has the potential to offer support for Xule in the future.


\section{EDGAR}
\label{sec:edgar}
% The SEC (US \texttt{S}ecurities and \texttt{E}xchange \texttt{C}ommission) maintains a system called EDGAR \cite{sec_edgar}. \footnote{not to be confused with the U.S. SEC, which stands for U.S. Soybean Export Council}
% EDGAR is a database of financial reports submitted to the SEC.
% EDGAR stands for \textbf{E}lectronic \textbf{D}ata \textbf{G}athering, \textbf{A}nalysis, and \textbf{R}etrieval.
% Filings submitted to EDGAR have to be in XBRL format.
% The SEC provides public access to the filings submitted to EDGAR. \footnote{\url{https://www.sec.gov/edgar/search/}}
The SEC, known as the U.S. Securities and Exchange Commission, 
operates a system referred to as EDGAR (Electronic Data Gathering, Analysis, and Retrieval) \cite{sec_edgar}. 
This EDGAR system serves as a public repository for XBRL reports submitted to the SEC\footnote{\url{https://www.sec.gov/edgar/search/}}.
% To be accepted by EDGAR, filings must be in the XBRL format. 
% The SEC offers public access to the filings stored within EDGAR\footnote{\url{https://www.sec.gov/edgar/search/}}.

\section{ESEF filings}
% The european counterpart to the SEC is called ESMA (\textbf{E}uropean \textbf{S}ecurities and \textbf{M}arkets \textbf{A}uthority).
% It also maintains a database of ESEF filings\cite{esma_database}.
% ESEF stands for \textbf{E}uropean \textbf{S}ingle \textbf{E}lectronic \textbf{F}ormat,
% which is a standard for XBRL reports in the EU.
% Similar to EDGAR, the ESEF database is publicly accessible. \footnote{\url{https://filings.xbrl.org/}}
% An interesting aspect of this database is that it is hosted by XBRL International, the organization that maintains the XBRL standard.
The European counterpart to the SEC is ESMA, 
the European Securities and Markets Authority, 
which operates a database containing ESEF filings \cite{esma_database}. 
ESEF stands for European Single Electronic Format, 
a standardized format for XBRL reports within the European Union.

Much like EDGAR, the ESEF database is accessible to the public\footnote{\url{https://filings.xbrl.org/}}. 
An intriguing aspect of this database is its hosting by XBRL International, the organization responsible for maintaining the XBRL standard.

\section{SEC - Interactive Data Public Test Suite}
\label{sec:idpts}
% To ensure that the XBRL processors that the companies use to create their XBRL reports are compliant with the XBRL standard, the SEC created the Interactive Data Public Test Suite\cite{sec_idpts}.
% This test suite contains an extensive collection of XBRL reports that are used to test XBRL processors.
% The SEC provides this test suite free of charge and it is available on their website.
In order to verify the compliance of XBRL processors utilized by companies for generating XBRL reports, 
the SEC established the Interactive Data Public Test Suite \cite{sec_idpts}. 
This comprehensive test suite comprises a vast assortment of XBRL reports designed to assess the performance of XBRL processors.
It is noteworthy that the SEC offers this test suite at no cost, and it can be accessed on their official website.

% Brel uses this test suite to test its XBRL processor.
% At the time of writing, Brel does not pass all the tests in the test suite, since Brel does not support all the features of the XBRL standard.
% However, building a processor that passes all the tests is not a feat that can feasibly be accomplished in the scope of this thesis.
% Brel employs the aforementioned test suite to evaluate its XBRL processor. 
% Currently, Brel does not successfully pass all the tests contained within the suite, 
% primarily due to its lack of support for certain features within the XBRL standard.
Brel does not employ the aforementioned test suite to evaluate its XBRL processor.
% Since this suite contains test cases which are tailored to EDGAR specific features,
% Most of the test cases in the suite cover features that are either specific to EDGAR or features that Brel does not support currently.
The majority of the test cases within the suite focus on functionalities that are unique to EDGAR or on features that Brel presently does not support.

\section{ESMA Conformance Suite}
% The European Securities and Markets Authority (ESMA) also maintains a test suite for XBRL processors\cite{esma_conformance_suite}.
% This test suite is similar to the SEC's Interactive Data Public Test Suite, but it is maintained by a different authority.
% Another key difference is that the ESMA Conformance Suite facilitates automated testing of xHTML reports.

% Brel does not support xHTML reports, so the ESMA Conformance Suite is not relevant for this thesis.
% However, Brel should be able to support xHTML reports in the future.
ESMA, the European Securities and Markets Authority, 
administers a test suite designed for XBRL processors, as well \cite{esma_conformance_suite}. 
This test suite shares similarities with the SEC's Interactive Data Public Test Suite, albeit being under the administration of a different regulatory authority. 
One notable divergence is that the ESMA Conformance Suite is tailored to facilitate automated testing of xHTML reports.

As of the present, Brel does not possess the capability to support xHTML reports, 
rendering the ESMA Conformance Suite irrelevant to the scope of this thesis. 
Nonetheless, it is important to acknowledge that there are intentions for Brel to incorporate support for xHTML reports in the future.
