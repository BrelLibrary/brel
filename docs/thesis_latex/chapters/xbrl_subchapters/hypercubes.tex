\section{Hypercubes}
\label{sec:hypercubes}

One key observation that can be made when looking at the facts of an XBRL report is that they are often structured like a hypercube.
The aspects of a fact can be seen as the dimensions of a hypercube, whereas the value of the fact is the value of the hypercube at the given dimensions.
Unlike networks, hypercubes are part of the OIM
\footnote{Hypercubes are the reason why the OIM does not only cover the XBRL 2.1 core specification, but also the XBRL Dimensions 1.0 specification.}.

\begin{figure}[H]
    \caption{Example of a hypercube}
    % create a figure with side by side images
    % the first is a table representing a XBRL hypercube.
    % It has 3 dimensions: "Period", "Entity", and "Concept".
    % The second image is a 3D representation of the hypercube.
    \label{fig:example_hypercube}
    \subfigure{
        \begin{tabular}{|c|c|c|c|}
            \hline
            Period & Entity & Concept & Value \\
            \hline
            2020 & Foo & Sales & 100\$ \\
            \hline
            2020 & Foo & Costs & 50\$ \\
            \hline
            2020 & Bar & Sales & 200\$ \\
            \hline
            2020 & Bar & Costs & 100\$ \\
            \hline
            2021 & Foo & Sales & 150\$ \\
            \hline
            2021 & Foo & Costs & 75\$ \\
            \hline
            2021 & Bar & Sales & 250\$ \\
            \hline
            2021 & Bar & Costs & 125\$ \\
            \hline
        \end{tabular}
    }
    \subfigure{
        TODO: 3D image of hypercube
    %     \includegraphics[width=0.5\textwidth]{images/hypercube.png}
    }
\end{figure}

Hypercubes are a common way to structure data nowadays.
Yet, back when XBRL was created, they were not as prevalent as they are today.
In fact, the early versions of XBRL did not support hypercubes at all.
They were retrofitted into the XBRL specification in 2006.\cite{xbrl_dimensions}.

\subsection{Dimensions}

When viewing facts as hypercubes, the cube ends up having four built in dimensions.
These correspond to the four core aspects of a fact: \texttt{Period}, \texttt{Entity}, \texttt{Concept}, and \texttt{Unit}.
XBRL allows for the creation of custom dimensions, which come in two flavors: explicit and typed.

Unfortunately, XBRL overloads the term "dimension".
It refers to both the dimensions of a hypercube, as well as the two custom dimension types. 

\subsection{Explicit dimensions}

Explicit dimensions are dimensions that have a predefined set of possible values.
For example, let us assume that the Foo Company has two subsidiaries: Foo United States and Foo Europe.
The Foo Company could then create a dimension called \texttt{Subsidiary} with the two possible values \texttt{Foo United States} and \texttt{Foo Europe}.
The possible values of an explicit dimension are called \texttt{members}.

Both dimensions and members are defined using report elements, just like concepts and abstracts before them.
To symbolize that a member belongs to a dimension, the member is defined as a child of the dimension in the definition network.

The members of a dimension can also have even more child members themselves.
For example, the Subsidiary Foo Europe could have two subsidiaries: Foo Switzerland and Foo EU.

\begin{figure}[H]
    \label{fig:example_explicit_dimension}
    \caption{Visualizations of the explicit dimension "Subsidiary"}
    \dirtree{%
        .1 [Dimension] Subsidiary.
        .2 [Member] Foo United States.
        .2 [Member] Foo Europe.
        .3 [Member] Foo Switzerland.
        .3 [Member] Foo EU.
    }
\end{figure}

\subsection{Typed dimensions}

Typed dimensions are dimensions that do not have a predefined set of possible values.
Instead, the values of a typed dimension are constrained by a data type.
For example, a dimension could be constrained to only allow values of the type \texttt{xs:integer}.

Similar to explicit dimensions, typed dimensions are defined using report elements.
Unlike explicit dimensions, typed dimensions do not have members.
They consist solely of the dimension report element, which defines the data type of the dimension.

\subsection{Line items and hypercubes}

With our current understanding of hypercubes, we can only view the whole report as a single, gigantic hypercube.
Especially when considering the additional dimensions, most facts will use only a small subset of the possible dimensions.
This makes the resulting hypercube high dimensional, with most of the dimensions being unused.
Using the large hypercube as a basis for analysis would be very inefficient.
% This is rarely desirable, since the sheer size of the hypercube makes it hard to work with.

To solve this problem, XBRL introduces the \texttt{hypercube} report element.
Conceptually, a hypercube is a sub-hypercube of the whole report hypercube.
Hypercube report elements are usually defined an a per-role basis as part of a definition network.
It picks a subset of the dimensions of the whole report hypercube.
This subset is determined in the definition network, where "hypercube-dimension" arcs specify which dimensions are part of the hypercube.

Besides the hypercube report element, XBRL also introduces the \texttt{lineItems} report element.
LineItems are used to specify which concepts are part of the hypercube.
Reports can specify tens of thousands of concepts, but only a few of them are relevant for a particular role.
The LineItems report element specifies the relevant concepts by listing them as children in the definition network.

If understanding lineItems proves to be difficult, consider the following: 
LineItems are to concepts what dimensions are to members.

