\section{Concepts}
\label{sec:concepts}

In section \ref{sec:facts}, we learned that a fact is the smallest unit of information in an XBRL report. 
One of the core aspects of a fact is its concept. 
The concept refers to what is being reported. 

So for example, if a fact is reporting information about a company's revenue, then the concept of the fact is "Revenue".
In this section, we will take a closer look at concepts and how they are defined in XBRL.

Concepts are the fundamental building blocks of XBRL and they are defined in what is called the \textbf{taxonomy}.

\subsection{Taxonomy}

Simply put, the XBRL taxonomy is a collection of concepts. 
Each XBRL report defines its own taxonomy inside of a taxonomy schema file. 
The taxonomy defined by the report is called the \textbf{extension taxonomy}.

This extension taxonomy contains references to other taxonomies, which may contain references to even more taxonomies.
So when a report and its extension taxonomy are loaded into memory, the entire taxonomy is loaded into memory.
The transitive closure of all these references is called the \textbf{DTS} (short for \textbf{D}iscoverable \textbf{T}axonomy \textbf{S}et).

Note that most of the taxonomies in the DTS are not located on the same machine as the report.
Instead, they are located on the internet and are downloaded on demand.

Some taxonomies commonly found in a report's DTS are:

\begin{itemize}
    \item \textbf{us-gaap} \footnote{https://xbrl.us/us-gaap/} - Contains concepts for US Generally Accepted Accounting Principles (GAAP).
    \item \textbf{ifrs} \footnote{https://www.ifrs.org/} - Contains concepts for International Financial Reporting Standards (IFRS).
    \item \textbf{dei} \footnote{https://www.sec.gov/info/edgar/dei-2019xbrl-taxonomy} - Contains concepts for the SEC's Document and Entity Information (DEI) requirements.
    \item \textbf{country} \footnote{https://xbrl.fasb.org/us-gaap/2021/elts/us-gaap-country-2021-01-31.xsd} - Contains concepts for country codes.
    \item \textbf{iso4217} \footnote{https://www.iso.org/iso-4217-currency-codes.html} - Contains concepts for currency codes.
\end{itemize}

Since a lot of DTSs from different reports share a lot of the same taxonomies, it makes sense to cache the taxonomies locally.

\subsection{Concepts}

Each concept in the DTS is identified by a \textbf{QName}. 
The technical intrecacies of QNames will be covered in section \ref{sec:qnames}, but for now think of them as a unique identifier for a concept.
The QName of a concept tends to be human-readable and self-explanatory. 
However, accountants and business analysts tend to go overboard with with their naming conventions.
Some examples of the QNames of concepts are:

\begin{itemize}
    \item \texttt{us-gaap:Assets}
    \item \texttt{ko:IncrementalTaxAndInterestLiability}
    \item \texttt{dei:EntityCommonStockSharesOutstanding}
    \item \texttt{us-gaap:ElementNameAndStandardLabelInMaturityNumericLowerEndToNumericHigherEndDateMeasureMemberOrMaturityGreaterThanLowEndNumericValueDateMeasureMemberOrMaturityLessThanHighEndNumericValueDateMeasureMemberFormatsGuidance}
\end{itemize}

But concepts do not only consist of a QName.
They also constrain some of the aspects and values of the facts that reference them.
Going back to our running example from section \ref{sec:facts}, the concept \texttt{us-gaap:Revenue} introduces some constraints.

For example, it constrains the value to be a \texttt{monetaryItemType}.
This means that the value of the fact must be a number and not just any arbitrary string.\footnote{The \texttt{monetaryItemType} has additional constraints besides being an integer, but we will ignore them for now.}
It also constrains the unit to be a currency defined in the ISO 4217 standard.\cite{eba2018filingrules}
Moreover, monetary facts have to be labeled as either "debit" or "credit" using the \texttt{balance} aspect.
In this case, the concept constrains the fact to have a "debit" balance, since the company's revenue is an asset.

The concept \texttt{us-gaap:Revenue} also constrains the period of the fact to be of type "duration".
This means that the period of the fact has to be a duration of time, such as a fiscal year or a quarter.
Alternatively, the period could be of type "instant", which means that the period refers to a specific point in time.

Finally, the concept \texttt{us-gaap:Revenue} allows the fact to be null, which means that there does not have to be a fact for the concept \texttt{us-gaap:Revenue}.
The vast majority of concepts allow facts to be null.

This wraps up our discussion about concepts in XBRL.
In the next section, we will take a look at QNames and how they are used in XBRL.
Together with the knowledge about concepts and facts, this will give us a solid foundation to understand the core parts of the XBRL standard.