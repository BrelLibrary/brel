\section{Concepts}
\label{sec:concepts}

In section \ref{sec:facts}, we learned that a fact is the smallest unit of information in an XBRL report. 
% One of the core aspects of a fact is its concept,
% which refers to what is being reported.
The central aspect of a fact is its concept,
which details the subject of the reported information.
% So for example, if a fact is reporting information about a company's revenue, then the concept of the fact is "Revenue".
% In this section, we will take a closer look at concepts and how they are defined in XBRL.
% Concepts are the fundamental building blocks of XBRL and they are defined in what is called the \textbf{taxonomy}.
For instance, if a fact conveys data regarding a company's revenue, then "Revenue" is the concept associated with this fact.
This section aims to delve deeper into concepts and their specification within XBRL.
Concepts are essential components of XBRL, outlined within what is known as the \textbf{taxonomy}.

\subsection{Taxonomy}

% Simply put, the XBRL taxonomy is a collection of concepts and their relationships.
% It is different from the XBRL instance document, which contains the actual facts of a report. 
% Each XBRL report defines its own taxonomy inside of a taxonomy schema file. 
% The taxonomy defined by the report is called the \textbf{extension taxonomy}.
In essence, the XBRL taxonomy is a collection of concepts and the relationships between them.
It differs from the XBRL instance document, which holds the report's actual data in the form of facts.
Each XBRL report outlines its taxonomy within a taxonomy schema file,
referred to as the \textbf{extension taxonomy}.

This extension taxonomy contains references to other taxonomies, which, in turn, may link to additional taxonomies.
Hence, when a report and its extension taxonomy are loaded into memory, the entire span of referenced taxonomies is also loaded.
The transitive closure of all these references is called the \textbf{DTS} (short for Discoverable Taxonomy Set).

% Note that most of the taxonomies in the DTS are not located on the same machine as the report.
% Instead, they are located on the internet and are downloaded on demand.

% Some taxonomies commonly found in a report's DTS are:

It's important to note that most taxonomies within the DTS are not stored on the same machine as the instance document and the extension taxonomy.
Rather, they are hosted online and fetched when required.

Commonly encountered taxonomies in a report's DTS include 
\texttt{us-gaap}\footnote{https://xbrl.us/us-gaap/}, which contains concepts for US Generally Accepted Accounting Principles (GAAP),
\texttt{dei}\footnote{https://www.sec.gov/info/edgar/dei-2019xbrl-taxonomy}, which contains concepts for the SEC's Document and Entity Information (DEI) requirements,
\texttt{iso4217}\footnote{https://www.iso.org/iso-4217-currency-codes.html}, which contains concepts for currency codes, 
and many more.

% Since a lot of DTSs from different reports share a lot of the same taxonomies, it makes sense to cache the taxonomies locally, 
% instead of downloading them every time they are needed.

% \subsection{Concepts}

% Each concept in the DTS is identified by a \textbf{QName}. 
% The technical intricacies of QNames will be covered in section \ref{sec:qnames}, but for now think of them as a unique identifier for a concept.
% The QName of a concept tends to be human-readable and self-explanatory. 
% However, accountants and business analysts tend to go overboard with with their naming conventions.
% Some examples of the QNames of concepts are:

Given that many DTSs from different reports tend to share taxonomies, it is advisable to store these taxonomies locally,
rather than re-downloading them each time they are required.

\subsection{Concepts}

Within the DTS, every concept is designated by a \textbf{QName},
% The complexities of QNames will be explored in section \ref{sec:qnames}. 
which will be discussed in detail in section \ref{sec:qnames}.
% For the moment, consider them as a unique identifier for each concept.
For the moment, the reader can think of them as a unique identifier for each concept.
Typically, the QName of a concept is designed to be both human-readable and self-explanatory.
Nevertheless, it's common for accountants and business analysts to employ elaborate naming conventions.
Here are some examples of QNames for concepts, all of which are extracted from Coca-Cola's 2019 Q2 report\cite{ko2019q2}:

\begin{itemize}
    \item \texttt{us-gaap:Assets}
    \item \texttt{ko:IncrementalTaxAndInterestLiability}
    \item \texttt{dei:EntityCommonStockSharesOutstanding}
    \item \texttt{us-gaap:ElementNameAndStandardLabelInMaturityNumericLowerEndTo-} 
    
    \texttt{NumericHigherEndDateMeasureMemberOrMaturityGreaterThanLowEnd-}
    
    \texttt{NumericValueDateMeasureMemberOrMaturityLessThanHighEndNumeric-}
    
    \texttt{ValueDateMeasureMemberFormatsGuidance}
\end{itemize}

% But concepts do not only consist of a QName.
% They also constrain some of the aspects and values of the facts that reference them.
% Going back to our running example from section \ref{sec:facts}, the concept \texttt{us-gaap:Revenue} introduces some constraints.
Concepts in XBRL extend beyond merely possessing a QName.
They also impose restrictions on certain aspects and values of the facts that reference these concepts.
Using our ongoing example from section \ref{sec:facts}, the concept \texttt{us-gaap:Revenue} sets forth specific constraints.

% For instance, it restricts the value to a \texttt{monetaryItemType}.
It restricts the value to a \texttt{monetaryItemType},
% This designation mandates that the fact's value must be numerical,
which mandates that the fact's value must be numerical,
as opposed to any arbitrary string.\footnote{The \texttt{monetaryItemType} encompasses additional restrictions beyond being numerical, but these will be set aside for now.}
It further limits the unit to currencies recognized by the ISO 4217 standard.\cite{eba2018filingrules}
Moreover, monetary facts must be identified as either "debit" or "credit" through the \texttt{balance} attribute.
In the context of \texttt{us-gaap:Revenue}, this stipulation means the fact should reflect a "debit" balance, attributing revenue as an asset.

% For example, it constrains the value to be a \texttt{monetaryItemType}.
% This means that the value of the fact must be a number, 
% not just any arbitrary string.\footnote{The \texttt{monetaryItemType} has additional constraints besides being an integer, but we will ignore them for now.}
% It also constrains the unit to be a currency defined in the ISO 4217 standard.\cite{eba2018filingrules}
% Moreover, monetary facts have to be labeled as either "debit" or "credit" using the \texttt{balance} attribute.
% In this case, the concept \texttt{us-gaap:Revenue} constrains the fact to have a "debit" balance, since the company's revenue is an asset.

% The concept \texttt{us-gaap:Revenue} also constrains the period of the fact to be of type "duration".
% This means that the period of the fact has to be a duration of time, such as a fiscal year or a quarter.
% Alternatively, the period could be of type "instant", which means that the period refers to a specific point in time.
The concept \texttt{us-gaap:Revenue} also specifies that the fact's period should be of the "duration" type.
This indicates that the period must span a certain time frame, such as a fiscal year or quarter,
% or it could be of the "instant" type, referring to a specific moment in time.
opposed to being of the "instant" type, which would denote a specific moment in time.

% Finally, the concept \texttt{us-gaap:Revenue} allows the fact to be null, 
% which means that it is optional for a report to contain a fact for the concept \texttt{us-gaap:Revenue}.
% The vast majority of concepts allow facts to be null.

% This wraps up our discussion about concepts in XBRL.
% In the next section, we will take a look at QNames and how they are used in XBRL.
% Together with the knowledge about concepts and facts, this will give us a solid foundation to understand the core parts of the XBRL standard.
This concludes our exploration of concepts in XBRL.
The forthcoming section will delve into QNames and their role within XBRL,
complementing our understanding of concepts and facts to solidify our grasp of XBRL's fundamental components.