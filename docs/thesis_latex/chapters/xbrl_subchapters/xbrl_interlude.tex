\section{Segway to advanced XBRL}

This concludes our overview of QNames as well as the overview of the core concepts of XBRL.
Armed with this knowledge, we could already create functional XBRL reports, albeit with a lot of limitations.

The most glaring limitation is that the facts in the report are not structured in any way.
% They are just a bunch of facts that are not related to each other.
Reports are just an unstructured set of facts.
Since facts are not related to each other, it is impossible to verify if the values within the report are consistent.

Besides that, XBRL in its current form is not very user-friendly.
The concepts for example are named using QNames, which tend to be human-readable but extremely verbose.
Another limitation is that QNames are mostly in English, which makes it difficult for non-English speakers to understand the report.

All of these issues will be addressed in the next sections, which will introduce the advanced concepts of XBRL, 
the most important of which are networks.
