% \section{Segway to advanced XBRL}
\section{Transition to Advanced XBRL Concepts}

% This concludes our overview of QNames as well as the overview of the core concepts of XBRL.
% Armed with this knowledge, we could already create functional XBRL reports, albeit with a lot of limitations.
With the completion of our discussion on QNames, 
we have set the groundwork necessary for generating functional, albeit limited, XBRL reports.
% However, these reports would be subject to significant constraints.


% The most glaring limitation so far is that the facts in the report are not structured in any way.
% Reports are just an unstructured set of facts.
% Since facts are not related to each other, it is impossible to verify if the values within the report are consistent.
A notable limitation at this stage is the lack of structure among the facts in a report, 
resulting in a collection of facts without any inherent organization. 
% This unstructured nature hinders the ability to verify the consistency of values within the report, as the facts are not interrelated.
Since facts are not interrelated, it is impossible to verify if the values within the report are consistent.

% Besides that, XBRL in its current form is not very user-friendly.
% The concepts for example are named using QNames, which tend to be human-readable but extremely verbose.
% Another limitation is that QNames are mostly in English, which makes it difficult for non-English speakers to understand the report.
Furthermore, the current state of XBRL is not particularly user-friendly. 
For instance, the use of QNames for naming concepts, while generally human-readable, results in verbosity. 
% Additionally, the predominance of English in QNames poses challenges for non-English speakers in comprehending the report.
Additionally, the predominantly English nature of QNames poses challenges for non-English speakers in understanding the report.

% All of these issues will be addressed in the next sections, which will introduce the advanced concepts of XBRL, 
% the most important of which are networks.
% All sections so far have covered the OIM, whereas the next sections will delve into the non-OIM parts of XBRL.
The upcoming sections aim to address these challenges by introducing the more sophisticated aspects of XBRL, with a focus on networks. 
These advanced topics extend beyond the Open Information Model (OIM) and delve into the more traditional, XML-based aspects of XBRL.

