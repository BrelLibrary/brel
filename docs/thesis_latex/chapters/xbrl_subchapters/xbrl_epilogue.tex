\section{XBRL Summary}

% As we have seen in this chapter, XBRL is a complex standard with many moving parts.
% It is a standard that has been in development for over 20 years, and it shows.
As demonstrated in this chapter, XBRL is a multifaceted standard with numerous components,
which has been refined over a period of more than 20 years.

% In the first half of this chapter, we have seen how an XBRL report is, at its core, just a collection of facts.
% The second half of this chapter has shown us how these facts can be structured into networks, roles and hypercubes.
% This chapter was in no way exhaustive, and there are many more aspects of XBRL that we have not covered.
% However, the aspects that we have covered are the ones that are most relevant to this thesis.
% The chapter was also, as mentioned in the introduction, 
% heavily based on both the XBRL standard\cite{xbrl} created by Charles Hoffman and 
% Dr. Ghislain Fourny's interpretation of it in \textit{the XBRL Book}\cite{fourny2023xbrl}.
The initial portion of this chapter explained that an XBRL report fundamentally comprises a set of facts.
Subsequently, the latter section illustrated how these facts can be organized into networks, roles, and hypercubes.
% While this chapter wasn't exhaustive, it focused on the aspects most pertinent to this thesis.
While this chapter was not exhaustive, it concentrated on the aspects most relevant to this thesis.

The succeeding chapter will introduce the API of Brel, 
% and how it interprets the XBRL standard.
illustrating how Brel interprets the XBRL standard.
The API chapter precedes the implementation chapter, which will explain how Brel implements the XBRL standard.
% Even though the API is part of the result of this thesis, it makes more sense to introduce the API before its implementation.
% Despite the API being a result of this thesis, introducing it before its implementation aligns logically.
Despite being a result of this thesis, it is more logical to introduce the API before its implementation.
