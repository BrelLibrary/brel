\section{QNames}
\label{sec:qnames}

% Although the motivation behind the XBRL processor Brel is to shield its user from the complexity of XML, 
% we keep one key aspect of XML in our API: QNames.

% QNames are a way to uniquely identify an XML element or attribute. 
% They consist of three things: a namespace prefix, a namespace URI, and a local name.
% The prefix acts as a shorthand for the namespace URI.

% For example the QName \texttt{us-gaap:Assets} identifies the element \texttt{Assets} in the namespace \texttt{us-gaap}.

% In this example, the namespace prefix \texttt{us-gaap} is a shorthand for the namespace URI \texttt{https://xbrl.fasb.org/us-gaap/2022/elts/us-gaap-2022.xsd}, 
% and together they form the namespace \texttt{us-gaap}.

Even though Brel aims to simplify the user's interaction by abstracting away the complexity of XML,
it retains a crucial element of XML within its API: QNames.

QNames serve as unique identifiers for XML elements or attributes,
comprising three components: a namespace prefix, a namespace URI, and a local name.
The prefix is a shorthand representation of the namespace URI, which we will refer to as a namespace binding.

For instance, the QName \texttt{us-gaap:Assets}, as defined by the Financial Accounting Standards Board (FASB)\cite{fasb},
signifies the element \texttt{Assets} within the \texttt{us-gaap} namespace.
% In this example, the namespace prefix \texttt{us-gaap} stands in for the URI \texttt{https://xbrl.fasb.org/us-gaap/2022/elts/us-gaap-2022.xsd}\cite{fasb}.
In this example, the namespace prefix \texttt{us-gaap} is bound to the URI \texttt{https://xbrl.fasb.org/us-gaap/2022/elts/us-gaap-2022.xsd}\cite{fasb}.

% \begin{figure}[H]
%     % \caption{The us-gaap:Assets QName}
%     \label{fig:qname_us_gaap_assets}
%     \begin{itemize}
%         \item Namespace prefix: \texttt{us-gaap}
%         \item URI: \texttt{https://xbrl.fasb.org/us-gaap/2022/elts/us-gaap-2022.xsd}
%         \item Local name: \texttt{Assets}
%     \end{itemize}
% \end{figure}

% QNames are used in XBRL to identify concepts, facts and other elements. Since they provide a robust and easy way for identifying elements,
Since QNames offer a robust and straightforward method for identifying elements,
% we decided to use them in our API as well. 
Brel employs them in its API.
However, there is one important difference between QNames in Brel and QNames in XBRL:
Currently, most XBRL filings are based on XML, where namespace bindings are defined on a per-element basis.
% Therefore, the mapping from namespace prefixes to namespace URIs depends on where the QName is used. 
% Therefore, the mapping of namespace prefixes forms a hierarchical structure, 
% where the mapping from namespace prefixes to namespace URIs depends on the location of the QName.
Therefore, the namespace bindings form a hierarchical structure,
where the namespace binding of a QName depends on the location of the QName.

% In our API, there is a fixed, global mapping from namespace prefixes to namespace URIs.
% The motivation behind this decision is that it makes the API easier to use.
% More details about this mapping will be explained in section \ref{sec:qnames_implementation}.
Brel takes a different approach, employing a fixed, global mapping from namespace prefixes to namespace URIs.
This decision was made to simplify the API.
Further details about this mapping will be provided in section \ref{sec:qnames_implementation}.

% Example of us-gaap and how it points to 2021 at one point and to 2022 at another point

% \hrule

% \textbf{Example 1:} The following XML document contains two \texttt{us-gaap:Assets} elements, each with a different namespace URI.

% \begin{lstlisting}
%     <?xml version="1.0" encoding="UTF-8"?>
%     <root xmlns:us-gaap="https://xbrl.fasb.org/us-gaap/2022/elts/us-gaap-2022.xsd">
    
%       <us-gaap:Assets>
%         <!-- Content for the first us-gaap:Assets element with XSD 2022 -->
%         <us-gaap:AssetValue>1000000</us-gaap:AssetValue>
%       </us-gaap:Assets>
    
%       <!-- Overwrite the namespace URI for the following element -->
%       <us-gaap:Assets xmlns:us-gaap="https://xbrl.fasb.org/us-gaap/2022/elts/us-gaap-2021.xsd">
%         <!-- Content for the second us-gaap:Assets element with XSD 2021 -->
%         <us-gaap:AssetValue>750000</us-gaap:AssetValue>
%       </us-gaap:Assets>
    
%     </root>
% \end{lstlisting}

% In this example, the first \texttt{us-gaap:Assets} element maps to the 2022 version of the US-GAAP taxonomy, while the second \texttt{us-gaap:Assets} element maps to the 2021 version of the US-GAAP taxonomy.
% This is possible because the namespace prefix \texttt{us-gaap} is redefined in the second \texttt{us-gaap:Assets} element.

% \hrule

% \textbf{Example 2:} Example that defines the same concept twice, once with 2021 and once with 2022.
