\subsection{Networks}

Up to this point, all concepts in the XBRL taxonomy are treated as independent entities.
The whole filing can be viewed as an unordered set of facts with no relation between them.
However, in reality, concepts are not independent of each other.
Instead, concepts are often related to each other in some way.

For example, the concepts "Assets" and "Liabilities" are related to each other in the sense that they are both part of the concept "Balance Sheet".
Furthermore, the concept "Assets" can be further divided into the concepts "Current Assets" and "Non-Current Assets".
The afforementioned relations can be visualized using a directed graph, as shown in figure \ref{fig:example_visualization_network_xbrl}.

\begin{figure}[H]
    \caption{Example of a relations between concepts in XBRL}
    \label{fig:example_visualization_network_xbrl}
    \dirtree{%
        .1 Balance Sheet.
        .2 Assets.
        .3 Current Assets.
        .3 Non-Current Assets.
        .2 Liabilities.
    }
\end{figure}

Any reader with a basic understanding of computer science will recognize that the above example is a graph, more specifically a tree.
Graphs are commonly used to represent relations between entities.
In the context of XBRL, graphs are used to represent all kinds of relations between concepts, facts, and other elements of an XBRL filing.
From now on, I will refer to graphs that represent relations between concepts as "networks".
XBRL commonly refers uses the term "Extended Link" to refer to networks or parts of networks.

\subsection{Types of Networks}

The XBRL 2.1 specification defines 6 built in types of networks\cite{xbrl21_terminology}\footnote[1]{The XBRL 2.1 specification is inconsistent about \texttt{link:footNotelink}. Section 1.4 does not list it as a standard extended link, section 3.5.2.4 does. I will assume that it is a standard extended link.}:

\begin{itemize}
    \item \texttt{link:presentationLink}: A network that represents the hierarchy of concepts in a report. An example of this can be seen in figure \ref{fig:example_visualization_network_xbrl}.
    \item \texttt{link:calculationLink}: A network that represents how concepts are calculated from other concepts. 
    For example, in figure \ref{fig:example_visualization_network_xbrl}, the concept "Assets" is calculated as the sum of the concepts "Current Assets" and "Non-Current Assets".
    \item \texttt{link:definitionLink}: A network that represents the relations between concepts and their definitions. 
    For example, there might be a concept "TotalAssets" that has the same semantic meaning as the concept "Assets". 
    A definition network would represent that the two concepts are aliases of each other.
    \item \texttt{link:labelLink}: A network that associates report elements with human-readable labels.
    \item \texttt{link:referenceLink}: A network that links report elements to external resources. 
    For example, the concept "Total Shareholder Return Amount" might have an official definition in the SEC's Code of Federal Regulations (CFR).
    The reference network would link the concept to the subparagraph \texttt{17 CFR 229.402(v)(2)(iv)}\cite{cfr_total_shareholder_return_amount}.
    \item \texttt{link:footnoteLink}: A network that links report elements, facts and other elements to footnotes.
\end{itemize}

XBRL refers to these built in networks as "standard extended links". 
If needed, XBRL allows users to define their own networks, which are referred to as "custom extended links"\cite{xbrl21_terminology}.

In the subsequent sections, I will describe how networks are implemented in XBRL on a conceptual level. 
