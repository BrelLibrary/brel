\label{sec:roles}
\section{Roles}

Even though the networks introduced in the previous section \ref{sec:xbrl_networks} provide a good foundation for structuring XBRL reports, 
they are not sufficient to create a comprehensive overview of the report whole report, only individual sections of it.
Moreover, the roll-up consistency of calculation networks \ref{sec:roll_up_consistency} introduced the notion of having networks related to each other.
With our current understanding of XBRL, there is no way to express this relationship.
This is precisely the function of \texttt{Roles} in XBRL.
% This is where \texttt{Roles} come into play. 

% Roles are a way to group networks together into a what is essentially a chapter of a report.
% Each set of networks is assigned a unique URI and potentially a label.

% For example, a report might have a role for the cover page, one for the balance sheet, one for the income statement, and so on.
% The balance sheet would only contain a presentation network, while the income statement would contain a presentation network, a calculation network, and potentially a definition network.

% A role usually contains a presentation network, a calculation network, and a definition network.
% The other types of networks are not commonly used in roles.
% Rather, they belong to the report as a whole.
% An example of this would be a label network that contains all the labels for the entire report.

% Roles in the XBRL XML syntax follow a simple structure, which I will explain using an example of a balance sheet role.

% The networks outlined in the previous section \ref{sec:xbrl_networks} establish a solid framework for structuring sections of XBRL reports. 
% However, they fall short in offering a cohesive view of an entire report, focusing instead on its segmented parts. 
% The concept of roll-up consistency within calculation networks \ref{sec:roll_up_consistency} hinted at the interconnection between networks, 
% suggesting a need for a mechanism to express these relationships. This is precisely the function of \texttt{Roles} in XBRL.

Roles serve to group networks, akin to chapters within a report, by assigning each network set a unique URI and, optionally, a label. 
This structuring allows for a segmented yet unified report presentation, where each section, such as the cover page, balance sheet, 
and income statement, is encapsulated within a specific role. 
Typically, a role encompasses a presentation network, a calculation network, and a definition network, 
with these networks forming the core components of a report section. 
Other network types, such as label and reference networks, are usually associated with the report as a whole rather than being confined to specific roles.

The balance sheet section, for example, might solely consist of a presentation network, whereas the income statement could integrate a presentation network, 
a calculation network, and possibly a definition network, reflecting the complexity and requirements of each report section.

The implementation of roles within the XBRL XML syntax\cite{xbrl21_custom_roles} adopts a straightforward approach, 
which will be explained through an example of a balance sheet role.

\begin{figure}[H]
    \begin{lstlisting}[language=XML,basicstyle=\small\ttfamily]
<link:roleType 
  id="BalanceSheet" 
  roleURI="http://www.foocompany.com/role/BalanceSheet"
>
    <link:definition>Foo balance</link:definition>
    <link:usedOn>link:presentationLink</link:usedOn>
    <link:usedOn>link:calculationLink</link:usedOn>
</link:roleType>
\end{lstlisting}
\caption{Example of the role "Balance Sheet" expressed in XBRL XML syntax}
\label{fig:example_role_xbrl}
\end{figure}

% The role in figure \ref{fig:example_role_xbrl} has the following properties:

% \begin{itemize}
%     \item \texttt{roleURI} (required): The URI of the role. This URI is used to reference the role from other elements in the XBRL taxonomy. It is the primary identifier of the role.
%     \item \texttt{definition} (optional): A human-readable description of the role.
%     \item \texttt{usedOn}: A list of links that the role can be used in.
% \end{itemize}

% The networks that are associated with the role are not defined in the role itself.
% Rather, each link that uses the role has to declare the role in the \texttt{role} property,
% which is used to reference the role from the link.

% Whenever a link references a role, the role must have a \texttt{usedOn} property that contains the type of the link.
% Going back to figure \ref{fig:example_role_xbrl}, if a definition network would reference the balance sheet role, 
% a conformant XBRL processor would throw an error.
% This is because the balance sheet role does not declare the \texttt{definitionLink} type in its \texttt{usedOn} property.

Figure \ref{fig:example_role_xbrl} showcases a role with certain attributes:

\begin{itemize}
    \item \texttt{roleURI} (required): The unique URI for the role. Other elements within the XBRL taxonomy utilize this URI to link to the role. It serves as the role's primary identifier.
    \item \texttt{definition} (optional): An accessible explanation of the role's purpose.
    \item \texttt{usedOn}: Specifies the types of links the role is applicable for.
\end{itemize}

Associations between networks and the role do not reside within the role itself.
Instead, link elements employing the role must specify it in the \texttt{role} attribute to establish the connection.

A link element that includes a role necessitates the role's \texttt{usedOn} attribute to list the link's type.
Referring again to figure \ref{fig:example_role_xbrl}, 
% an error would be generated by a compliant XBRL processor if a definition network attempted to use the balance sheet role. 
a compliant XBRL processor would throw an error if a definition network attempted to reference the balance sheet role.
This error occurs because the balance sheet role lacks a \texttt{definitionLink} designation in its \texttt{usedOn} list.

% \vspace{5cm}