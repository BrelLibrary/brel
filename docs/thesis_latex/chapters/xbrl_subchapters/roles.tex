\label{sec:roles}
\section{Roles}

Even though the networks introduced in the previous section \ref{sec:xbrl_networks} provide a good foundation for structuring XBRL reports, 
they are not sufficient to create a comprehensive overview of the report whole report, only individual sections of it.
Moreover, the roll-up consistency of calculation networks\ref{sec:roll_up_consistency} introduced the notion of having networks related to each other.
With our current understanding of XBRL, there is no way to express this relationship.
This is where \texttt{Roles} come into play. 

Roles are a way to group networks together into a what is essentially a chapter of a report.
Each set of networks is assigned a unique URI and potentially a label.
% They fill in the gap of ...

For example, a report might have a role for the cover page, one for the balance sheet, one for the income statement, and so on.
The balance sheet would only contain a presentation network, while the income statement would contain a presentation network, a calculation network, and potentially a definition network.

A role usually contains a presentation network, a calculation network, and a definition network.
The other types of networks are not commonly used in roles.
Rather, they belong to the report as a whole.
An example of this would be a label network that contains all the labels for the entire report.

Roles in the XBRL XML syntax follow a simple structure, which I will explain using an example of a balance sheet role.

% \begin{figure}[H]
%     \caption{Example of the component "Inventory" in XBRL from Tesla Inc.'s 2023 Q3 report\cite{tesla_10q_2023_q3}}
%     \label{fig:example_component_xbrl}
%     % make the text size smaller
%     \begin{lstlisting}[language=XML,basicstyle=\fontsize{7}{10}\selectfont\ttfamily]
%       <link:roleType id="Inventory" roleURI="http://www.tesla.com/role/Inventory">
%         <link:definition>0000010 - Disclosure - Inventory</link:definition>
%         <link:usedOn>link:presentationLink</link:usedOn>
%         <link:usedOn>link:calculationLink</link:usedOn>
%         <link:usedOn>link:definitionLink</link:usedOn>
%       </link:roleType>
%     \end{lstlisting}
% \end{figure}
\begin{figure}[H]
    \caption{Example of the role "Balance Sheet" expressed in XBRL XML syntax}
    \label{fig:example_role_xbrl}
    \begin{lstlisting}[language=XML,basicstyle=\fontsize{7}{10}\selectfont\ttfamily]
        <link:roleType id="BalanceSheet" roleURI="http://www.foocompany.com/role/BalanceSheet">
            <link:definition>Foo balance</link:definition>
            <link:usedOn>link:presentationLink</link:usedOn>
            <link:usedOn>link:calculationLink</link:usedOn>
        </link:roleType>
    \end{lstlisting}
\end{figure}

The role in figure \ref{fig:example_role_xbrl} has the following properties:

\begin{itemize}
    \item \texttt{roleURI} (required): The URI of the role. This URI is used to reference the role from other elements in the XBRL taxonomy. It is the primary identifier of the role.
    \item \texttt{definition} (optional): A human-readable description of the role.
    \item \texttt{usedOn}: A list of links that the role can be used in.
\end{itemize}

The networks that are associated with the role are not defined in the role itself.
Rather, each link that uses the role has to declare the role in the \texttt{role} property,
which is used to reference the role from the link.

Whenever a link references a role, the role must have a \texttt{usedOn} property that contains the type of the link.
Going back to figure \ref{fig:example_role_xbrl}, if a definition network would reference the balance sheet role, 
a conformant XBRL processor would throw an error.
This is because the balance sheet role does not declare the \texttt{definitionLink} type in its \texttt{usedOn} property.