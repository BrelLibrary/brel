\section{Facts}

A fact is the smallest unit of information in an XBRL report. 
The word "Fact" is a term used to describe an individual piiece of financial of business information within an XBRL instance document. 

Lets consider a simplified example involving a financial report for a company. 
Suppose the report contains information about the company's revenue for the fiscal year ending on December 31, 2022.

In XBRL, a corresponding fact would be represented as follows:

\begin{itemize}
    \item \textbf{Concept:} Revenue
    \item \textbf{Entity:} Microsoft Coorporation
    \item \textbf{Period:} from 2022-04-01 to 2023-03-31 \footnote[0]{Refers to the fiscal year 2022, which starts on April 1, 2022 and ends on March 31, 2023}
    \item \textbf{Unit:} USD
    \item \textbf{Value:} 198'000'000 \footnote[1]{https://www.microsoft.com/investor/reports/ar22/index.html}
\end{itemize}

In this example:

\begin{itemize}
    \item The \textbf{concept} refers \textit{what} is being reported. 
    In this case, "Revenue" indicates that the fact is reporting information about the company's revenue.
    \item The \textbf{entity} refers to \textit{who} is reporting. 
    In the case of our example, the entity is "Microsoft Coorporation".
    \item The \textbf{period} refers to \textit{when} the information is being reported.
    The period is defined as the fiscal year 2022.
    \item The \textbf{unit} refers to \textit{how} the information is being reported.
    In this example, the unit is "USD", which indicates that the information is being reported in US dollars.
    \item The \textbf{value} refers to \textit{how much} is being reported.
    According to Microsoft's 2022 annual report, the company's revenue for the fiscal year 2022 was around 198 billion US dollars.
\end{itemize}

The concept, entity, period and unit of a fact are called its \textbf{aspects}. 
If necessary, additional aspects can be defined for a fact. 
These additional aspects are called \textbf{dimensions} and will be covered in section TODO.
The aspects that are not dimensions are called \textbf{core aspects}. 
Even though the name suggests otherwise, core aspects are not all mandatory for a fact.
The only mandatory core aspect is the concept.