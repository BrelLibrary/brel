\section{Usability}
\label{sec:usability}

% One of the goals of this thesis is to create a usable API for working with XBRL reports.
% To evaluate the usability of the Brel API, we will implement a simple CLI XBRL report viewer.
% This viewer will cover every feature of the Brel API and serve as a proof of concept for the Brel API.

% The CLI XBRL report viewer will be implemented in Python and will use the Brel API to load and display XBRL reports.
% It will be able to load XBRL reports from local files and from URLs.

% The viewer will be able to display the following information about an XBRL report:
A primary aim of this thesis is to develop a user-friendly API for XBRL report processing.  
For assessing the Brel API's usability, a basic Command Line Interface (CLI) XBRL report viewer will be created.  
This viewer will showcase a subset of all features of the Brel API, demonstrating its practicality.

The CLI XBRL report viewer, coded in Python, will employ the Brel API for XBRL report management.  
It will have the capability to access XBRL reports both locally and via URLs.
The viewer is available as an example in the Brel repository\cite{brel_source}.

Brel is, first and foremost, a wrapper around the XBRL standard.
Even though its main goal is not to visualize XBRL reports, 
it still provides a handful of methods that enable visualization of XBRL reports in the console.
% The viewer will use these methods to display the following information about an XBRL report:
However, these methods were not covered in chapter \ref{chapter:api}, 
since they were built on top of the core API and merely serve as convenience methods.
Nonetheless, this section will give a brief explanation of how these methods are implemented.
\pagebreak

Regarding an XBRL report, the viewer will present various types of information:

\begin{itemize}
    \item The facts in the report, which can be filtered by concept and a dimension\footnote{Filters for entity, period, and unit are not implemented since most reports have very few entities, periods, and units.}.
    % For each fact, the viewer should display all characteristics of the fact in a easy-to-read manner.
    For each fact, the viewer displays all characteristics in a human-readable format.
    \item The components of a report together with their networks.
    % The viewer should be able to display the relationships between report elements in the networks in a human-readable manner.
    The viewer presents the relationships between report elements in the networks in a human-readable format.
\end{itemize}

Before implementing the viewer, we will first install Brel and its dependencies.
Since Brel is published on the Python Package Index (PyPI), we can install it using pip.

\begin{lstlisting}[language=bash]
pip install brel-xbrl
\end{lstlisting}

After installing Brel, we can start implementing the viewer.
The viewer will be implemented in a single file called \texttt{viewer.py}.
The first part of the implementation is shown in figure \ref{fig:viewer_1}.
It shows the imports and the implementation of the CLI responsible for user input.
\begin{figure}[H]
    \centering
    \lstinputlisting[
        language=Python, 
        basicstyle=\scriptsize\ttfamily,
        firstline=1, lastline = 8
    ]{../../examples/cli.py}
    \caption{The implementation of the CLI XBRL report viewer responsible for user input.}
    \label{fig:viewer_1}
\end{figure}


% The viewer is implemented as a command-line interface (CLI) using the \texttt{argparse} module.
The viewer uses the \texttt{argparse} module for user input. 
% It has two subcommands: \texttt{facts} and \texttt{components}.
The CLI has two subcommands: \texttt{facts} and \texttt{components}.
% Users can utilize the former is used to display the facts in an XBRL report,
Users can use the former to display the facts in an XBRL report, 
% and the \texttt{components} subcommand is used to display the components of a report together with their networks.
whereas the latter subcommand is used to display the components together with their networks.
Both subcommands take a single argument, which acts as a filter for the facts and components that are displayed.

% The facts filter is used to filter the facts in the report by concept or dimension.
% If a fact has the concept or dimension that matches the filter, it will be displayed.
% The components filter is used to filter the components in the report by URI.
% If a component has a URI that contains the filter as a substring, it will be displayed
% \footnote{We use a substring since typing out the full URI of a component can be cumbersome.}.
The report's facts filter selects facts based on matching concepts or dimensions.
A fact is displayed if it aligns with the filter's specified concept or dimension.
The components filter operates by screening report components via their URI.
A component is shown if its URI includes the filter's substring\footnote{Substrings are used to simplify the process, as typing the full URI can be cumbersome.}.

% First of all, the viewer uses \texttt{Filing.open} on line 11 to open the XBRL report.
% The argument of this method can be a local file path or a URI.
% Since Brel potentially needs to download the report from the internet, the call to \texttt{Filing.open} can take a few seconds to complete.
% The figure \ref{fig:viewer_2} shows the second part of the implementation of the viewer responsible for loading the report.
Initially, the viewer employs \texttt{Filing.open} to load the XBRL report.
This method's argument can be either a local file path or a URI.
Due to the potential need for Brel to download the report, executing \texttt{Filing.open} might take a few seconds.
The following figure illustrates the viewer's second part, focusing on the report's loading process.

\begin{figure}[H]
    \centering
    % \caption{The implementation of the XBRL report viewer responsible for loading the report}
    \lstinputlisting[
        language=Python, 
        basicstyle=\scriptsize\ttfamily,
        firstline=11, lastline = 12
    ]{../../examples/cli.py}
    \label{fig:viewer_2}
    \caption{The code responsible for loading the XBRL report.}
\end{figure}

% The facts portion of the viewer \ref{fig:viewer}, which ranges from lines 12 to 23, uses the Brel methods \texttt{report.get\_all\_facts}, 
% \texttt{fact.get\_concept}, \texttt{fact.get\_aspects}, and \texttt{utils.pprint}.
% Some of the methods were not covered in chapter \ref{chapter:api} since they are merely convenience methods that are not part of the core API.
% For example, \texttt{fact.get\_concept} is an alias of \texttt{fact.get\_characteristic(Aspect.CONCEPT)}.
% Furthermore, the \texttt{utils.pprint} method is a convenience method that pretty-prints most Brel objects in a human-readable manner.
% It simply calls both \texttt{fact.get\_aspects} and \texttt{fact.get\_characteristic} for each fact and pretty-prints the result.
The next section of the implementation outlines the facts portion of the viewer.
First, it gets all facts in the report.
Then, it filters the facts that either have the concept or dimension that matches the filter.
Next, it pretty-prints the facts in a human-readable manner.
We can get all facts in the report using the method \texttt{report.get\_all\_facts}.
The \texttt{fact.get\_concept} and \texttt{fact.get\_aspects} methods can be used to get the concept and aspects of each fact.
Since dimensions are just aspects, we can check all aspects of a fact to see if one of them matches the filter.
Finally, we can use the \texttt{utils.pprint} method to pretty-print the facts in a human-readable manner.
\texttt{utils.pprint} is a convenience method that, given a list of facts, it calls both \texttt{fact.get\_aspects} and \texttt{fact.get\_characteristic} for each fact and pretty-prints the result.
Figure \ref{fig:viewer_3} shows the third part of the implementation of the viewer responsible for displaying the facts in the report.

\begin{figure}[H]
    \centering
    % \caption{The implementation of the XBRL report viewer responsible for displaying the facts in the report}
    \lstinputlisting[
        language=Python, 
        basicstyle=\scriptsize\ttfamily,
        firstline=13, lastline = 21
    ]{../../examples/cli.py}
    \caption{The code responsible for displaying the facts in the XBRL report.}
    \label{fig:viewer_3}
\end{figure}

% The components portion of the viewer should be implemented in a similar manner.
% First, it should get all components in the report using \texttt{report.get\_all\_components}.
% Then, it should filter the components that have a URI that contains the filter as a substring.
% This is done using the \texttt{component.get\_uri} method.
% Finally, it should pretty-print the components in a human-readable manner using \texttt{utils.pprint}.
% The method \texttt{utils.pprint} not only works for lists of facts, but also for a list components.
% For each network in the component, the \texttt{utils.pprint} uses a DFS algorithm on both the
% \texttt{network.get\_roots} and \texttt{network\_node.get\_children} methods to pretty-print the network in a human-readable manner.
% It also automatically uses the labels of report elements if they are available.
% The figure \ref{fig:viewer_4} shows the fourth and final part of the implementation of the viewer responsible for displaying the components in the report.
% The viewer's components section should be implemented following a similar approach.  
The viewer's components section is implemented in a similar manner.
Initially, it needs to retrieve all components in the report by using \texttt{report.get\_all\_components}.  
Next, it filters those components whose URI contains the specified substring, utilizing \texttt{component.get\_uri} for this purpose.  
Finally, \texttt{utils.pprint} is used to format the components in a human-readable manner.
The \texttt{utils.pprint} method is versatile, functioning with both fact lists and component lists.  
For each network within a component, \texttt{utils.pprint} employs a depth-first search algorithm,  
applying it to both \texttt{network.get\_roots} and \texttt{network\_node.get\_children} to format the network clearly and coherently.  
This method also automatically incorporates the labels of report elements when available.  
Figure \ref{fig:viewer_4} displays the final part of the viewer's implementation, focusing on presenting the report's components.


\begin{figure}[H]
    \centering
    % \caption{The implementation of the XBRL report viewer responsible for displaying the components in the report}
    \lstinputlisting[
        language=Python, 
        basicstyle=\scriptsize\ttfamily,
        firstline=22, lastline = 34
    ]{../../examples/cli.py}
    \caption{The code responsible for displaying the components in the XBRL report.}
    \label{fig:viewer_4}
\end{figure}

The implementation of the viewer is now complete.
The viewer can be used to display the facts and components of an XBRL report using only a few lines of code and using python's built-in list comprehension.

% As shown in figure \ref{fig:viewer}, filtering the facts in a report can be done using only a few lines of code and using python's built-in list comprehension.
% The method \texttt{report.get\_all\_facts} is used to get all facts in the report, 
% where the methods \texttt{fact.get\_concept} and \texttt{fact.get\_aspects} are used to get the concept and aspects of each fact and filter accordingly.

% The components portion of the viewer \ref{fig:viewer}, which ranges from lines 24 to 34,
% uses the Brel methods \texttt{report.get\_all\_components}, \texttt{component.get\_uri} and \texttt{utils.pprint}.
% The \texttt{utils.pprint} not only works for lists of facts, but also for a list components.
% For each network in the component, the \texttt{utils.pprint} uses a DFS algorithm on both the
% \texttt{network.get\_roots} and \texttt{network\_node.get\_children} methods to pretty-print the network in a human-readable manner.
% It also automatically uses the labels of report elements if they are available.
% Similar to the facts portion, the \texttt{get\_all\_components} method is used to get all components in the report,
% where the \texttt{component.get\_uri} method is used to get the URI of each component and filter accordingly.

% Again, as shown in figure \ref{fig:viewer}, filtering the components in a report can be done using only a few lines of code and using python's built-in list comprehension.


% For instance, the command \texttt{python viewer.py facts us-gaap:Revenue} will display all facts in the report that have the concept \texttt{us-gaap:Revenue}.
% The command \texttt{python viewer.py components CoverPage} will display all components that have the substring \texttt{CoverPage} in their URI.
% \begin{lstlisting}[language=Python, caption={The implementation of the CLI XBRL report viewer}, label={lst:viewer}]

The following two examples illustrate how the viewer can be used to display the facts and components of an XBRL report.
Each example will show both the command that is used to display the information and the output of the command.
For both examples, we will use the Q3 2023 10-Q report of Apple Inc. that is available on the SEC's website\cite{aapl_10q_2023_q3}.
% For brevity, we have downloaded the report and saved in an archive called \texttt{report.zip}.
To keep the commands simple, we will use the \texttt{report.zip} file that contains the report.
However, the viewer can also load reports from URLs.

\begin{figure}[H]
    \centering
    \begin{lstlisting}[language=bash]
python viewer.py report.zip --facts us-gaap:Assets
\end{lstlisting}
    % make font tiny and use monospaced font
    \begin{lstlisting}[language=bash, basicstyle=\scriptsize\ttfamily]
--------------------------------------------------------------------------------------------+
|                                    Facts Table                                            |
+----------------+---------------+------------------------------------+------+--------------+
|        concept |        period |                             entity | unit |        value |
+----------------+---------------+------------------------------------+------+--------------+
| us-gaap:Assets | on 2023-07-01 | {http://www.sec.gov/CIK}0000320193 |  usd | 335038000000 |
| us-gaap:Assets | on 2022-09-24 | {http://www.sec.gov/CIK}0000320193 |  usd | 352755000000 |
+----------------+---------------+------------------------------------+------+--------------+
\end{lstlisting}
    \caption{Assets in the Q3 2023 10-Q report of Apple Inc.}
    \label{fig:aapl_assets}
\end{figure}

As shown in figure \ref{fig:aapl_assets}, the command returns the facts in the report that have the concept \texttt{us-gaap:Assets}.
It clearly shows the concept, period, entity, unit, and value for both facts.
Neither fact has additional dimensions.

The \texttt{entity} column contains a QName that is a reference to the entity in the report.
In this case, \texttt{0000320193} is the Central Index Key (CIK) of Apple Inc
and \texttt{http://www.sec.gov/CIK} is the namespace of the CIK, which can be used to look up the entity in the report.

\begin{figure}[H]
    \begin{lstlisting}[language=bash]
python viewer.py report.zip --components InsiderTradingArrangements
\end{lstlisting}
    % make font tiny and use monospaced font. allow non-utf8 characters
    \begin{lstlisting}[basicstyle=\scriptsize\ttfamily, language=bash, extendedchars=true, escapeinside={\<}{\>}]
Component: http://xbrl.sec.gov/ecd/role/InsiderTradingArrangements
Info: 995445 - Disclosure - Insider Trading Arrangements
Networks: 
link role: .../InsiderTradingArrangements, link name: link:presentationLink
arc roles: ['.../parent-child'], arc name: link:presentationArc
└──[LINE ITEMS] Insider Trading Arrangements [Line Items]
    ├──[HYPERCUBE] Trading Arrangements, by Individual [Table]
    │  ├──[DIMENSION] Trading Arrangement [Axis]
    │  │  └──[MEMBER] All Trading Arrangements [Member]
    │  └──[DIMENSION] Individual [Axis]
    │     └──[MEMBER] All Individuals [Member]
    ├──[CONCEPT] Material Terms of Trading Arrangement [Text Block]
    │   ...
    └──[CONCEPT] Trading Arrangement, Securities Aggregate Available Amount
\end{lstlisting}
\caption{Insider trading arrangements in the Q3 2023 10-Q report of Apple Inc. The output is truncated.}

\label{fig:aapl_insider_trading}
\end{figure}

% As shown in figure \ref{fig:aapl_insider_trading}, 
% the command returns the component identified with URI \texttt{http://xbrl.sec.gov/ecd/role/InsiderTradingArrangements}.
% It clearly shows the URI, label, and networks for the component.
% The networks are pretty-printed in a human-readable manner and 
% show the relationships between report elements in the networks.
% Additionally, the network shows the link/arc roles and names for each network, which can be useful for debugging.

% This example shows that the Brel API is easy to use and can be used to implement a simple CLI XBRL report viewer.
% The viewer is able to display the facts and components using only a few lines of code and using python's built-in list comprehension.
% This section does not cover every feature of the Brel API, but it serves as a proof of concept.
% More examples are available both in the Brel documentation\cite{brel_api} and in the Brel repository\cite{brel_source}.

As depicted in figure \ref{fig:aapl_insider_trading},  
the command retrieves the component with the URI \texttt{http://xbrl.sec.gov/ecd/role/InsiderTradingArrangements}.  
Figure \ref{fig:aapl_insider_trading} shows the URI, label, and networks associated with the component.  
These networks are formatted in a user-friendly manner,  
illustrating the connections between various report elements within the networks.  
Furthermore, each network includes the link/arc roles and names, providing valuable insights for debugging purposes.

This example demonstrates the user-friendliness and practical utility of the Brel API.  
It exemplifies how a simple Command Line Interface (CLI) XBRL report viewer can be implemented using the Brel API.  
The viewer efficiently displays facts and components with minimal coding, leveraging Python's built-in list comprehension.  
While this section does not explore all features of the Brel API, it effectively serves as a proof of concept.  
For additional examples and in-depth understanding, readers can refer to the Brel documentation\cite{brel_api} and the Brel source code repository\cite{brel_source}.

% \end{lstlisting}